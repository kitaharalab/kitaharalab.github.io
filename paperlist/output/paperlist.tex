
\documentstyle[a4j,11pt]{jarticle}

\newcounter{savedenumi}
\newenvironment{Enumerate}
{\begin{enumerate} \setcounter{enumi}{\thesavedenumi} 
\setlength{\itemsep}{7pt} }
{\setcounter{savedenumi}{\theenumi} \end{enumerate}}

\title{研究業績
  
}
\author{北原 鉄朗}
\date{}

\begin{document}
\maketitle


\section*{学位論文}
\begin{Enumerate}
  
\item 
\underline{北原 鉄朗}: 
    ``Computational Musical Instrument Recognition and Its Application to Content-based Music
      Information Retrieval'', 
    博士論文 京都大学大学院情報学研究科, February 2007. 
{\bf (第2回京都大学総長賞受賞)}
\end{Enumerate}

\section*{学術論文}
\begin{Enumerate}
  
\item 
\underline{北原 鉄朗}, 
後藤 真孝, 
奥乃 博: 
    ``音高による音色変化に着目した楽器音の音源同定:F0依存多次元正規分布に基づく識別手法'', 
    {\it 情報処理学会論文誌,
    } Vol.44, No.10, pp.2448--2458, October 2003. 
{\bf (電気通信普及財団 第19回テレコムシステム技術学生賞 受賞)}
\item 
\underline{北原 鉄朗}, 
後藤 真孝, 
奥乃 博: 
    ``音響的類似性を反映した楽器の階層表現の獲得とそれに基づく未知楽器のカテゴリーレベルの音源同定'', 
    {\it 情報処理学会論文誌,
    } 特集「音楽情報科学」, Vol.45, No.3, pp.680--689, March 2004. 

\item 
石田 克久, 
\underline{北原 鉄朗}, 
武田 正之: 
    ``N-gramによる旋律の音楽的適否判定に基づいた即興演奏支援システム'', 
    {\it 情報処理学会論文誌,
    } 特集「インタラクション:技術と展開」, Vol.46, No.7, pp.1549--1559, July 2005. 

\item 
\underline{Tetsuro Kitahara}, 
Masataka Goto, 
and 
Hiroshi
      G. Okuno: 
    ``Pitch-dependent Identification of Musical Instrument Sounds'', 
    {\it Applied Intelligence,
    } Vol.23, No.3, pp.267--275, December 2005. 

\item 
藤原
      弘将, 
\underline{北原 鉄朗}, 
後藤
      真孝, 
駒谷
      和範, 
尾形 哲也, 
奥乃 博: 
    ``伴奏音抑制と高信頼度フレーム選択に基づく楽曲の歌手名同定手法'', 
    {\it 情報処理学会論文誌,
    } 特集「情報処理技術のフロンティア」, Vol.47, No.6, pp.1831--1843, July 2006. 

\item 
\underline{北原 鉄朗}, 
後藤
      真孝, 
駒谷
      和範, 
尾形 哲也, 
奥乃 博: 
    ``多重奏を対象とした音源同定: 混合音テンプレートを用いた音の重なりに頑健な特徴量の重みづけ および音楽的文脈の利用'', 
    {\it 電子情報通信学会論文誌,
    } Vol.J89-D, No.12, pp.2721--2733, December 2006. 

\item 
\underline{Tetsuro Kitahara}, 
Masataka Goto, 
Kazunori Komatani, 
Tetsuya
      Ogata, 
and 
Hiroshi
      G. Okuno: 
    ``Instrument Identification in Polyphonic Music: Feature Weighting to Minimize Influence of
      Sound Overlaps'', 
    {\it EURASIP Journal on Advances in Signal Processing,
    } Special Issue on Music Information Retrieval based on Signal Processing, Vol.2007, No.51979, pp.1--15, 2007. 

\item 
\underline{Tetsuro Kitahara}, 
Masataka Goto, 
Kazunori Komatani, 
Tetsuya
      Ogata, 
and 
Hiroshi
      G. Okuno: 
    ``Instrogram: Probabilistic Representation of Instrument Existence for Polyphonic Music'', 
    {\it IPSJ Journal,
    } Special Issue on Convenient, Familiar Music Information Processing, Vol.48, No.1, pp.214--226, January 2007. 
{\bf (第3回IPSJ Digital Courier船井若手奨励賞)}(also published in IPSJ Digital Courier Vol.3, No.1, pp.1--13)
\item 
\underline{北原 鉄朗}, 
勝占 真規子, 
片寄 晴弘, 
長田 典子: 
    ``ベイジアンネットワークを用いた自動コードヴォイシングシステム'', 
    {\it 情報処理学会論文誌,
    } 特集「音楽情報処理」, Vol.50, No.3, pp.1067--1078, March 2009. 

\item 
橋田 光代, 
松井 淑恵, 
\underline{北原 鉄朗}, 
片寄 晴弘: 
    ``ピアノ名演奏の演奏表現情報と音楽構造情報を対象とした音楽演奏表情データベースCrestMusePEDBの構築'', 
    {\it 情報処理学会論文誌,
    } 特集「音楽情報処理」, Vol.50, No.3, pp.1090--1099, March 2009. 

\item 
Hiromasa Fujihara, 
Masataka Goto, 
\underline{Tetsuro Kitahara}, 
and 
Hiroshi
      G. Okuno: 
    ``Singing Voice Representation Robust to Accompaniment Sounds and Its Application to Singer
      Identification and Vocal-timbre-similarity-based Music Information Retrieval'', 
    {\it IEEE Transaction on Audio, Speech, and Language Processing,
    } Special Issue on Signal Models and Representation of Musical and Environmental Sounds, Vol.18, No.3, pp.638--648, March 2010. 

\end{Enumerate}

\section*{ショートペーパー}
\begin{Enumerate}
  
\item 
石田 克久, 
\underline{北原 鉄朗}, 
武田 正之: 
    ``N-gramによる即興演奏の旋律補正'', 
    {\it 情報処理学会論文誌(テクニカルノート),
    } 特集「音楽情報科学」, Vol.45, No.3, pp.743--746, March 2004. 

\item 
\underline{北原 鉄朗}, 
戸谷 直之, 
徳網 亮輔, 
片寄 晴弘: 
    ``BayesianBand:ユーザとシステムが相互に予測し合うジャムセッションシステム'', 
    {\it 情報処理学会論文誌(テクニカルノート),
    } 特集「エンターテインメントコンピューティング」, Vol.50, No.12, pp.2949--2953, December 2010. 

\end{Enumerate}

\section*{国際会議}
\begin{Enumerate}
  
\item 
\underline{Tetsuro Kitahara}, 
Masataka Goto, 
and 
Hiroshi
      G. Okuno: 
    ``Musical Instrument Identification based on F0-dependent Multivariate Normal Distribution'', 
    {\it Proceedings of
      the 2003 IEEE International Conference on Acoustics, Speech, and Signal Processing
          (ICASSP 2003),
        } Vol.V, pp.421--424, April 2003. 
(Cancelled because of SARS)
\item 
\underline{Tetsuro Kitahara}, 
Masataka Goto, 
and 
Hiroshi
      G. Okuno: 
    ``Pitch-dependent Musical Instrument Identification and Its Application to Musical Sound
      Ontology'', 
    {\it Developments in Applied Artificial Intelligence --- Proceedings of the 16th
      International Conference on Industrial Engineering Applications of Artificial Intelligence and
      Expert Systems (IEA/AIE-2003),
    } LNAI 2718, (P. W. H. Chung, C. Hinde and M. Ali (Eds.)), pp.112--122, Springer, July 2003. 

\item 
\underline{Tetsuro Kitahara}, 
Masataka Goto, 
and 
Hiroshi
      G. Okuno: 
    ``Musical Instrument Identification based on F0-dependent Multivariate Normal Distribution'', 
    {\it Proceedings of the 2003 IEEE International Conference
      on Multimedia \& Expo
          (ICME 2003),
        } Vol.III, pp.409--412, July 2003. 
(Reprint of the paper published in ICASSP 2003)
\item 
\underline{Tetsuro Kitahara}, 
Masataka Goto, 
and 
Hiroshi
      G. Okuno: 
    ``Acoustical-similarity-based Musical Instrument Hierarchy and Its Application to Musical
      Instrument Identification'', 
    {\it Proceedings of the 2004 International Symposium on
      Musical Acoustics
          (ISMA 2004),
        } 3-S2-12, pp.397--300, April 2004. 
(abstract reviewed)
\item 
\underline{Tetsuro Kitahara}, 
Masataka Goto, 
and 
Hiroshi
      G. Okuno: 
    ``Category-level Identification of Non-registered Musical Instrument Sounds'', 
    {\it Proceedings of
      the 2004 IEEE International Conference on Acoustics, Speech, and Signal Processing
          (ICASSP 2004),
        } Vol.IV, pp.253--256, May 2004. 

\item 
Yohei Sakuraba, 
\underline{Tetsuro Kitahara}, 
and 
Hiroshi
      G. Okuno: 
    ``Comparing Features for Forming Music Streams in Automatic Music Transcription'', 
    {\it Proceedings of
      the 2004 IEEE International Conference on Acoustics, Speech, and Signal Processing
          (ICASSP 2004),
        } Vol.IV, pp.273--376, May 2004. 

\item 
Katsuhisa
      Ishida, 
\underline{Tetsuro Kitahara}, 
and 
Masayuki Takeda: 
    ``ism: Improvisation Supporting System based on Melody Correction'', 
    {\it Proceedings of the International Conference on New
      Interfaces for Musical Expression 
          (NIME 2004),
        } pp.177--180, June 2004. 

\item 
Takuya Yoshioka, 
\underline{Tetsuro Kitahara}, 
Kazunori Komatani, 
Tetsuya
      Ogata, 
and 
Hiroshi
      G. Okuno: 
    ``Automatic Chord Transcription with Concurrent Recognition of Chord Symbols and Boundaries'', 
    {\it Proceedings of
      the 5th International Conference on Music Information Retrieval
          (ISMIR 2004),
        } pp.100--105, October 2004. 

\item 
\underline{Tetsuro Kitahara}, 
Masataka Goto, 
Kazunori Komatani, 
Tetsuya
      Ogata, 
and 
Hiroshi
      G. Okuno: 
    ``Instrument Identification in Polyphonic Music: Feature Weighting with Mixed Sounds,
      Pitch-dependent Timbre Modeling, and Use of Musical Context'', 
    {\it Proceedings of
      the 6th International Conference on Music Information Retrieval 
          (ISMIR 2005),
        } pp.558--563, September 2005. 

\item 
Hiromasa Fujihara, 
\underline{Tetsuro Kitahara}, 
Masataka Goto, 
Kazunori Komatani, 
Tetsuya
      Ogata, 
and 
Hiroshi
      G. Okuno: 
    ``Singer Identification based on Accompaniment Sound Reduction and Reliable Frame Selection'', 
    {\it Proceedings of
      the 6th International Conference on Music Information Retrieval 
          (ISMIR 2005),
        } pp.329--336, September 2005. 

\item 
\underline{Tetsuro Kitahara}, 
Katsuhisa
      Ishida, 
and 
Masayuki Takeda: 
    ``ism: Improvisation Supporting Systems with Melody Correction and Key Vibration'', 
    {\it Entertainment Computing --- Proceedings of the 4th International Conference on
      Entertainment Computing (ICEC 2005),
    } LNCS 3711, (F. Kishino, Y. Kitamura, H. Kato and N. Nagata (Eds.)), pp.315--327, September 2005. 

\item 
\underline{Tetsuro Kitahara}, 
Masataka Goto, 
Kazunori Komatani, 
Tetsuya
      Ogata, 
and 
Hiroshi
      G. Okuno: 
    ``Instrogram: A New Musical Instrument Recognition Technique without Using Onset Detection
      nor F0 Estimation'', 
    {\it Proceedings of
      the 2006 IEEE International Conference on Acoustics, Speech, and Signal Processing
          (ICASSP 2006),
        } Vol.V, pp.229--232, May 2006. 
{\bf (IEEE関西支部 第3回学生研究奨励賞 受賞)}
\item 
Hiromasa Fujihara, 
\underline{Tetsuro Kitahara}, 
Masataka Goto, 
Kazunori Komatani, 
Tetsuya
      Ogata, 
and 
Hiroshi
      G. Okuno: 
    ``F0 Estimation Method for Singing Voice in Polyphonic Audio Signal based on Statistical
      Vocal Model and Viterbi Search'', 
    {\it Proceedings of
      the 2006 IEEE International Conference on Acoustics, Speech, and Signal Processing
          (ICASSP 2006),
        } Vol.V, pp.253--256, May 2006. 

\item 
Hiromasa Fujihara, 
\underline{Tetsuro Kitahara}, 
Masataka Goto, 
Kazunori Komatani, 
Tetsuya
      Ogata, 
and 
Hiroshi
      G. Okuno: 
    ``Speaker Identification under Noisy Environments by using Harmonic Structure Extraction
      and Reliable Frame Weighting'', 
    {\it Proceedings of the International Conference on
      Spoken Language Processing
          (ICSLP 2006),
        } September 2006. 

\item 
Katsutoshi Itoyama, 
\underline{Tetsuro Kitahara}, 
Kazunori Komatani, 
Tetsuya
      Ogata, 
and 
Hiroshi
      G. Okuno: 
    ``Automatic Feature Weighting in Automatic Transcription of Specified Part in Polyphonic
      Music'', 
    {\it Proceedings of
      the 7th International Conference on Music Information Retrieval 
          (ISMIR 2006),
        } October 2006. 

\item 
\underline{Tetsuro Kitahara}, 
Masataka Goto, 
Kazunori Komatani, 
Tetsuya
      Ogata, 
and 
Hiroshi
      G. Okuno: 
    ``Musical Instrument Recognizer ``Instrogram'' and Its Application to Music Retrieval based
      on Instrumentation Similarity'', 
    {\it Proceesings of the 8th IEEE International Symposium on
      Multimedia
          (ISM 2006),
        } pp.265--272, December 2006. 

\item 
\underline{Tetsuro Kitahara}, 
Makiko Katsura, 
Haruhiro Katayose, 
and 
Noriko Nagata: 
    ``Computational Model for Automatic Chord Voicing based on Bayesian Network'', 
    {\it Proceedings of the 10th International Conference on
      Music Perception and Cognition
          (ICMPC 2008),
        } pp.395--398, August 2008. 

\item 
\underline{Tetsuro Kitahara}, 
Masahiro Nishiyama, 
and 
Hiroshi
      G. Okuno: 
    ``Computational Model of Congruency between Music and Video'', 
    {\it Proceedings of the 10th International Conference on
      Music Perception and Cognition
          (ICMPC 2008),
        } August 2008. 
(abstract only)
\item 
Mitsuyo Hashida, 
Teresa M. Nakra, 
Haruhiro Katayose, 
Tadahiro Murao, 
Keiji Hirata, 
Kenji Suzuki, 
and 
\underline{Tetsuro Kitahara}: 
    ``Rencon: Performance Rendering Contest for Automated Music Systems'', 
    {\it Proceedings of the 10th International Conference on
      Music Perception and Cognition
          (ICMPC 2008),
        } pp.53--57, August 2008. 

\item 
Yusuke Tsuchihashi, 
\underline{Tetsuro Kitahara}, 
and 
Haruhiro Katayose: 
    ``Using Bass-line Features for Content-based MIR'', 
    {\it Proceedings of
      the 9th International Conference on Music Information Retrieval 
          (ISMIR 2008),
        } pp.620--625, September 2008. 

\item 
\underline{Tetsuro Kitahara}, 
Yusuke Tsuchihashi, 
and 
Haruhiro Katayose: 
    ``Music Genre Classification and Similarity Calculation Using Bass-line Features'', 
    {\it Proceesings of the 10th IEEE
      International Symposium on Multimedia, Workshop on Multimedia Audio and Speech Processing
          (ISM 2008 MASP Workshop),
        } pp.574--579, December 2008. 

\item 
\underline{Tetsuro Kitahara}, 
Naoyuki Totani, 
Ryosuke Tokuami, 
and 
Haruhiro Katayose: 
    ``BayesianBand: Jam Session System based on Mutual Prediction by User and System'', 
    {\it Entertainment Computing: Proceedings of the 10th International Conference on
    Entertainment Computing (ICEC 2009),
    } pp.179--184, September 2009. 

\end{Enumerate}

\section*{国内査読付き会議}
\begin{Enumerate}
  
\item 
石田 克久, 
\underline{北原 鉄朗}, 
武田 正之: 
    ``ism:即興演奏の不自然な旋律を補正する演奏支援システム'', 
    {\it Proceedings of the 11th Workshop on Interactive
      Systems and Software
          (WISS 2003),
        } pp.19--24, December 2003. 

\item 
石田 克久, 
\underline{北原 鉄朗}, 
武田 正之: 
    ``演奏者に振動で情報提示する鍵盤楽器「ぶるぶるくん」'', 
    {\it Proceedings of the 12th Workshop on Interactive
      Systems and Software
          (WISS 2004),
        } pp.59--64, December 2004. 

\item 
三澤 由宇, 
細野 裕, 
仁科 章史, 
石田 克久, 
\underline{北原 鉄朗}, 
後藤 真孝, 
武田 正之: 
    ``Openism:旋律補正に基づく演奏支援機能付き遠隔地セッションシステム'', 
    {\it Proceedings of the 13th Workshop on Interactive
      Systems and Software
          (WISS 2005),
        } December 2005. 

\item 
戸谷 直之, 
\underline{北原 鉄朗}, 
片寄 晴弘: 
    ``楽器構成に着目した楽曲サムネイルとプレイリスト生成機能つき音楽プレイヤー'', 
    {\it インタラクション2008(インタラクティブ発表),
    } pp.173--174, March 2008. 

\item 
\underline{北原 鉄朗}, 
徳網 亮輔, 
戸谷 直之, 
橋本 寿政, 
片寄 晴弘: 
    ``BayesianBand:旋律の予測に基づいた自動伴奏システム'', 
    {\it インタラクション2009(インタラクティブ発表),
    } pp.31--32, March 2009. 

\end{Enumerate}

\section*{国内研究会}
\begin{Enumerate}
  
\item 
\underline{北原 鉄朗}, 
後藤 真孝, 
奥乃 博: 
    ``音高による音色変化に着目した音源同定手法'', 
    {\it 情報処理学会 音楽情報科学 研究報告,} 2001-MUS-40-2, Vol.2001, No.45, pp.7--14, May 2001. 

\item 
\underline{北原 鉄朗}, 
後藤 真孝, 
奥乃 博: 
    ``楽器音を対象とした音源同定:音高による音色変化を考慮する識別手法の検討'', 
    {\it 情報処理学会 音楽情報科学 研究報告,} 2002-MUS-46-1, Vol.2002, No.63, pp.1--8, July 2002. 

\item 
\underline{北原 鉄朗}, 
後藤 真孝, 
奥乃 博: 
    ``音響的特徴に基づく楽器の階層表現の獲得とそれに基づくカテゴリーレベルの楽器音認識の検討'', 
    {\it 情報処理学会 音楽情報科学 研究報告,} 2003-MUS-51-9, Vol.2003, No.82, pp.51--58, August 2003. 

\item 
吉井 和佳, 
\underline{北原 鉄朗}, 
櫻庭 洋平, 
奥乃 博: 
    ``自己組織化マップによる教師なしクラスタリングを利用したドラム演奏の自動採譜'', 
    {\it 情報処理学会 音楽情報科学 研究報告,} 2003-MUS-51-8, Vol.2003, No.82, pp.43--40, August 2003. 

\item 
後藤 真孝, 
平田 圭二, 
片寄 晴弘, 
平井 重行, 
濱中 雅俊, 
武田 晴登, 
\underline{北原 鉄朗}: 
    ``パネルディスカッション「音楽情報処理研究者\{に,が\}望むこと」'', 
    {\it 情報処理学会 音楽情報科学 研究報告,} 2003-MUS-51-5, Vol.2003, No.82, pp.25--28, August 2003. 

\item 
石田 克久, 
\underline{北原 鉄朗}, 
武田 正之: 
    ``ism:即興演奏支援のためのリアルタイム旋律補正システム'', 
    {\it 情報処理学会 ヒューマンインターフェース研究会/音楽情報科学研究会 研究報告,
    } 2003-HI-106-2, 2003-MUS-52-2, Vol.2003, No.111, pp.9--15, November 2003. 

\item 
\underline{北原 鉄朗}, 
後藤 真孝, 
奥乃 博: 
    ``TimbreTree:音色の類似度に基づいた楽器の階層的分類'', 
    {\it 日本音響学会 音楽音響研究会 資料,
    } MA2004-7, Vol.23, No.2, pp.13--18, June 2004. 

\item 
吉岡 拓也, 
\underline{北原 鉄朗}, 
尾形 哲也, 
奥乃 博: 
    ``音楽音響信号を対象とした和音進行の認識'', 
    {\it 日本音響学会 音楽音響研究会 資料,
    } MA2004-8, Vol.23, No.2, pp.19--24, June 2004. 

\item 
\underline{北原 鉄朗}, 
後藤 真孝, 
奥乃 博: 
    ``混合音テンプレートを用いた多重奏の音源同定'', 
    {\it 情報処理学会 音楽情報科学 研究報告,} 2004-MUS-56-9, Vol.2004, No.84, pp.57--64, August 2004. 

\item 
吉岡 拓也, 
\underline{北原 鉄朗}, 
尾形 哲也, 
奥乃 博: 
    ``和音区間検出と和音名同定の相互依存性を解決する和音認識手法'', 
    {\it 情報処理学会 音楽情報科学 研究報告,} 2005-MUS-56-6, Vol.2004, No.84, pp.33--40, August 2004. 

\item 
浜中 雅俊, 
\underline{北原 鉄朗}, 
石田 克久, 
谷井 章夫, 
竹川 佳成, 
吉井 和佳, 
宮下 芳明, 
上 田 健太郎: 
    ``デモンストレーション:若手による研究紹介'', 
    {\it 情報処理学会 音楽情報科学 研究報告,} 2004-MUS-56-6, Vol.2004, No.84, pp.27--32, August 2004. 

\item 
\underline{北原 鉄朗}, 
石田 克久, 
武田 正之: 
    ``振動機能付鍵盤楽器「ぶるぶるくん」を用いた即興演奏支援システム'', 
    {\it 情報処理学会 音楽情報科学 研究報告,} 2005-MUS-60-5, Vol.2005, No.45, pp.25--30, May 2005. 

\item 
浜中 雅俊, 
李 昇姫, 
池月 雄哉, 
石原 一志, 
\underline{北原 鉄朗}, 
野池 賢二, 
中野 倫靖, 
梶 克彦, 
岡 良典, 
平田 圭二, 
松田 周, 
青木 忍, 
上田 健太郎: 
    ``デモンストレーション:若手による研究紹介II'', 
    {\it 情報処理学会 音楽情報科学 研究報告,} 2005-MUS-61-5, Vol.2005, No.82, pp.27--33, August 2005. 

\item 
藤原
      弘将, 
\underline{北原 鉄朗}, 
後藤
      真孝, 
駒谷
      和範, 
尾形 哲也, 
奥乃 博: 
    ``伴奏音抑制と高信頼度フレーム選択に基づく楽曲中の歌声の歌手名同定手法'', 
    {\it 情報処理学会 音楽情報科学 研究報告,} 2005-MUS-61-16, Vol.2005, No.82, pp.97--104, August 2005. 

\item 
\underline{北原 鉄朗}, 
後藤
      真孝, 
駒谷
      和範, 
尾形 哲也, 
奥乃 博: 
    ``Instrogram: 発音時刻検出とF0推定の不要な楽器音認識手法'', 
    {\it 情報処理学会 音楽情報科学 研究報告,} 2006-MUS-66, Vol.2006, No.90, pp.69--76, August 2006. 

\item 
浜中 雅俊, 
竹川 佳成, 
橋田 朋子, 
元川 洋一, 
馬場 哲晃, 
日暮 圭, 
 中野 倫靖, 
吉井 和佳, 
松原 正樹, 
梶 克彦, 
\underline{北原 鉄朗}: 
    ``デモンストレーション:若手による研究紹介III'', 
    {\it 情報処理学会 音楽情報科学 研究報告,} 2006-MUS-66-10, Vol.2006, No.90, pp.55--61, August 2006. 

\item 
浜中 雅俊, 
竹川 佳成, 
岩井 憲一, 
高橋 直也, 
 中野 倫靖, 
大石 康智, 
糸山 克寿, 
\underline{北原 鉄朗}, 
吉井 和佳: 
    ``デモンストレーション:若手による研究紹介IV'', 
    {\it 情報処理学会 音楽情報科学 研究報告,} 2006-MUS-67-3, Vol.2006, No.113, pp.9--14, October 2006. 

\item 
西山 正紘, 
\underline{北原 鉄朗}, 
駒谷
      和範, 
尾形 哲也, 
奥乃 博: 
    ``マルチメディアコンテンツにおける音楽と映像の調和度計算モデル'', 
    {\it 情報処理学会 音楽情報科学 研究報告,} 2007-MUS-69, Vol.2007, No.15, pp.31--36, February 2007. 

\item 
安部 武宏, 
\underline{北原 鉄朗}, 
糸山 克寿, 
柳田 益造: 
    ``撥弦の物理モデルを用いた音響信号からのパラメータ推定'', 
    {\it 日本音響学会音楽音響研究会資料,
    } MA2006-91, pp.35--40, March 2007. 

\item 
\underline{北原 鉄朗}, 
橋田 光代, 
片寄 晴弘: 
    ``音楽情報科学研究のための共通データフォーマットの確立を目指して'', 
    {\it 情報処理学会 音楽情報科学 研究報告,} 2006-MUS-66-12, Vol.2007, No.81, pp.149--154, August 2007. 

\item 
平田 圭二, 
梶 克彦, 
亀岡 弘和, 
\underline{北原 鉄朗}, 
齋藤 毅, 
武田 晴登, 
橋田 光代: 
    ``新博士にょるパネルディスカッション1「博士への道のりと将来への夢」'', 
    {\it 情報処理学会 音楽情報科学 研究報告,} 2007-MUS-71-7, Vol.2007, No.81, pp.39--42, August 2007. 

\item 
\underline{北原 鉄朗}, 
後藤
      真孝, 
奥乃 博, 
片寄 晴弘: 
    ``楽器音認識技術を用いた音楽の可視化'', 
    {\it Proceedings of
      Entertainment Computing 2007
          (EC2007),
        } pp.145--148, October 2007. 

\item 
橋田 光代, 
松井 淑恵, 
\underline{北原 鉄朗}, 
酒造 祐介, 
片寄 晴弘: 
    ``音楽演奏表情データベースCrestMusePEDB ver1.0の公開について'', 
    {\it 情報処理学会 音楽情報科学 研究報告,} 2007-MUS-72-1, Vol.2007, No.102, pp.1--6, October 2007. 

\item 
土橋 佑亮, 
\underline{北原 鉄朗}, 
片寄 晴弘: 
    ``音響信号を対象としたベースラインからの音楽ジャンル解析'', 
    {\it 情報処理学会 音楽情報科学/音声言語情報処理 研究報告,
    } 2008-MUS-74-38, 2008-MUS-SLP-70-38, Vol.2008, No.12, pp.217--224, February 2008. 

\item 
勝占 真規子, 
\underline{北原 鉄朗}, 
片寄 晴弘, 
長田 典子: 
    ``ベイジアンネットワークを用いたコード・ヴォイシング推定システム'', 
    {\it 情報処理学会 音楽情報科学/音声言語情報処理 研究報告,
    } 2008-MUS-74-29, 2008-MUS-SLP-70-29, Vol.2008, No.12, pp.163--168, February 2008. 

\item 
藤田 徹, 
\underline{北原 鉄朗}, 
片寄 晴弘, 
長田 典子: 
    ``アーティストの個性を表す音楽的特徴に関する一考察'', 
    {\it 情報処理学会 音楽情報科学/音声言語情報処理 研究報告,
    } 2008-MUS-74-35, 2008-MUS-SLP-70-35, Vol.2008, No.12, pp.199--204, February 2008. 

\item 
後藤
      真孝, 
亀岡 弘和, 
\underline{北原 鉄朗}, 
平賀 譲, 
緒方 淳, 
戸田 智基: 
    ``パネルディスカッション「``音''研究の未来」'', 
    {\it 情報処理学会 音楽情報科学/音声言語情報処理 研究報告,
    } 2008-MUS-74-10, 2008-SLP-70-10, Vol.2008, No.12, pp.57--58, February 2008. 

\item 
\underline{北原 鉄朗}, 
小林 一樹, 
片寄 晴弘: 
    ``演奏家型人形を利用した見えない演奏者の可視化の試み'', 
    {\it インタラクション2008(ポスター発表),
    } March 2008. 

\item 
\underline{北原 鉄朗}, 
片寄 晴弘: 
    ``CrestMuseXML (CMX) Toolkit ver.0.40について'', 
    {\it 情報処理学会 音楽情報科学 研究報告,} 2008-MUS-75-17, Vol.2008 , No.50, pp.95--100, May 2008. 

\item 
\underline{北原 鉄朗}, 
平田 圭二, 
竹川 佳成, 
中野 倫靖, 
森勢 将雅, 
吉井 和佳: 
    ``新博士によるパネルディスカッションII「楽しくさせる音楽,楽しくさせる研究」'', 
    {\it 情報処理学会 音楽情報科学 研究報告,} 2008-MUS-76-1, Vol.2008 , No.78, pp.1--4, August 2008. 

\item 
橋本 祐輔, 
\underline{北原 鉄朗}, 
片寄 晴弘: 
    ``音楽音響信号を対象とした指揮演奏システム:フェルマータ時における打楽器音抑制とスケジューラの検討'', 
    {\it 情報処理学会 音楽情報科学 研究報告,} 2008-MUS-76-7, Vol.2008 , No.78, pp.33--37, August 2008. 

\item 
三浦 雅展, 
江村 伯夫, 
\underline{北原 鉄朗}, 
若槻 尚斗, 
藤島 琢哉, 
西口 磯春, 
平田 圭二, 
柳田 益造, 
後藤 真孝: 
    ``パネルディスカッション:作るだけでいいの?調べるだけでいいの?'', 
    {\it 情報処理学会 音楽情報科学 研究報告,} 2008-MUS-78-11, Vol.2008, No.78, pp.59--66, December 2008. 

\item 
橋田 光代, 
片寄 晴弘, 
平田 圭二, 
\underline{北原 鉄朗}, 
鈴木 健嗣: 
    ``演奏表情付けコンテストICMPC-Rencon開催報告'', 
    {\it 情報処理学会 音楽情報科学 研究報告,} 2008-MUS-78-12, Vol.2008, No.78, pp.67--72, December 2008. 

\item 
\underline{北原 鉄朗}: 
    ``CrestMuseXML Toolkitで始める音楽情報処理入門'', 
    {\it 情報処理学会 音楽情報科学 研究報告,
    } 2009-MUS-50-1, May 2009. 

\item 
戸谷 直之, 
\underline{北原 鉄朗}, 
片寄 晴弘: 
    ``予測型ジャムセッションシステムBayesianBandにおける可視化機能の導入'', 
    {\it エンターテインメントコンピューティング2009,
    } September 2009. 

\item 
橋田 光代, 
\underline{北原 鉄朗}, 
鈴木健嗣, 
平田 圭二, 
片寄 晴弘: 
    ``演奏表情付けコンテストEC-Rencon'', 
    {\it エンターテインメントコンピューティング2009,
    } September 2009. 

\item 
橋田 光代, 
\underline{北原 鉄朗}, 
鈴木 健嗣, 
片寄 晴弘, 
平田 圭二: 
    ``演奏表情付けコンテストEC-Rencon開催報告'', 
    {\it 情報処理学会 音楽情報科学 研究報告,
    } 2009-MUS-83, November 2009. 

\end{Enumerate}

\section*{国内全国大会}
\begin{Enumerate}
  
\item 
\underline{北原 鉄朗}, 
後藤 真孝, 
奥乃 博: 
    ``楽器音オントロジー作成のための楽器音特徴抽出'', 
    {\it 情報処理学会 第62回全国大会,} 4M-5, March 2001. 

\item 
柳川 貴央, 
\underline{北原 鉄朗}, 
武田 正之: 
    ``即興演奏における演奏補正システム'', 
    {\it 情報処理学会 第64回全国大会,} 1L-5, March 2002. 

\item 
\underline{北原 鉄朗}, 
後藤 真孝, 
奥乃 博: 
    ``音色空間の音高依存性を考慮した楽器音の音源同定'', 
    {\it 日本音響学会2002年秋季研究発表会 講演論文集,} 1-1-4, pp.643--644, September 2002. 

\item 
\underline{北原 鉄朗}, 
後藤 真孝, 
奥乃 博: 
    ``音響的類似性に基づく楽器音の階層的クラスタリング'', 
    {\it 情報処理学会 第64回全国大会,} 1P-1, March 2003. 
{\bf (学生奨励賞)}
\item 
吉井 和佳, 
\underline{北原 鉄朗}, 
櫻庭
      洋平, 
奥乃 博: 
    ``教師なしクラスタリングと認識誤りパターンを利用した打楽器音の音源同定'', 
    {\it 情報処理学会 第64回全国大会,} 1P-3, March 2003. 

\item 
石田 克久, 
\underline{北原 鉄朗}, 
柳川 貴央, 
奥乃 博: 
    ``統計的アプローチに基づく即興演奏補正'', 
    {\it 情報処理学会 第64回全国大会,} 1P-3, March 2003. 

\item 
\underline{北原 鉄朗}, 
後藤 真孝, 
奥乃 博: 
    ``未知の楽器を考慮する楽器音の音源同定'', 
    {\it 情報処理学会 第66回全国大会,} 3ZA-3, March 2004. 
{\bf (学生奨励賞)}
\item 
吉岡 拓也, 
吉井 和佳, 
\underline{北原 鉄朗}, 
櫻庭
      洋平, 
尾形 哲也, 
奥乃 博: 
    ``音楽音響信号を対象とした和音変化時刻と和音名の同時認識'', 
    {\it 情報処理学会 第66回全国大会,} 3ZA-4, March 2004. 

\item 
石田 克久, 
\underline{北原 鉄朗}, 
武田 正之: 
    ``統計モデルに基づく旋律妥当性判定手法を用いた即興演奏支援'', 
    {\it 日本音響学会2004年秋季研究発表会 講演論文集,} 2-6-8, pp.783--784, September 2004. 

\item 
\underline{北原 鉄朗}, 
後藤 真孝, 
駒谷
      和範, 
尾形 哲也, 
奥乃 博: 
    ``多重奏の音源同定のための混合音からのテンプレート作成法'', 
    {\it 情報処理学会 第67回全国大会,} 3G-4, March 2005. 
{\bf (大会奨励賞)}
\item 
藤原
      弘将, 
\underline{北原 鉄朗}, 
後藤
      真孝, 
尾形 哲也, 
奥乃 博: 
    ``歌声の調波構造抽出を用いた歌手名の同定'', 
    {\it 情報処理学会 第67回全国大会,} 3R-8, March 2005. 

\item 
\underline{北原 鉄朗}, 
後藤 真孝, 
駒谷
      和範, 
尾形 哲也, 
奥乃 博: 
    ``混合音からの特徴量テンプレート作成と音楽的文脈の利用による多重奏の音源同定'', 
    {\it 日本音響学会2005年秋季研究発表会 講演論文集,} 3-10-15, September 2005. 

\item 
海尻 聡, 
石原 一志, 
\underline{北原 鉄朗}, 
Valin Jean-Marc, 
駒谷
      和範, 
尾形 哲也, 
奥乃 博: 
    ``ロボットによる周囲状況把握のための雑音下での環境音認識'', 
    {\it 計測自動制御学会 第6回システムインテグレーション部門講演会 (SI2005),
    } December 2005. 

\item 
糸山 克寿, 
\underline{北原 鉄朗}, 
駒谷
      和範, 
尾形 哲也, 
奥乃 博: 
    ``多重奏中特定パートの自動採譜における複数特徴量の自動重み付け'', 
    {\it 情報処理学会 第68回全国大会,} 2L-6, March 2006. 

\item 
西山 正紘, 
\underline{北原 鉄朗}, 
駒谷
      和範, 
尾形 哲也, 
奥乃 博: 
    ``標題音楽アノテーションのための階層的物語タグの設計'', 
    {\it 情報処理学会 第68回全国大会,} 3L-6, March 2006. 

\item 
田口 明裕, 
\underline{北原 鉄朗}, 
石原 一志, 
駒谷
      和範, 
尾形 哲也, 
奥乃 博: 
    ``擬音語表現を利用した環境音のためのXMLタグの設計と自動付与'', 
    {\it 情報処理学会 第68回全国大会,} 3L-7, March 2006. 

\item 
\underline{北原 鉄朗}, 
後藤
      真孝, 
駒谷
      和範, 
尾形 哲也, 
奥乃 博: 
    ``Instrogram:楽器存在確率に基づく音楽視覚表現法'', 
    {\it 日本音響学会2006年春季研究発表会 講演論文集,} 2-2-13, March 2006. 

\item 
藤原
      弘将, 
\underline{北原 鉄朗}, 
後藤
      真孝, 
駒谷
      和範, 
尾形 哲也, 
奥乃 博: 
    ``調波構造抽出と高信頼度フレーム選択を用いた雑音下での話者識別'', 
    {\it 日本音響学会2006年春季研究発表会 講演論文集,} 1-11-17, March 2006. 

\item 
\underline{北原 鉄朗}, 
後藤
      真孝, 
駒谷
      和範, 
尾形 哲也, 
奥乃 博: 
    ``Instrogramを用いた類似楽曲検索'', 
    {\it 日本音響学会2006年秋季研究発表会 講演論文集,} 2-7-1, September 2006. 

\item 
西山 正紘, 
\underline{北原 鉄朗}, 
駒谷
      和範, 
尾形 哲也, 
奥乃 博: 
    ``マルチメディアコンテンツにおける音楽と映像の調和に関する分析'', 
    {\it 情報処理学会 第70回全国大会,} 2N-6, March 2007. 

\item 
清水 敬太, 
\underline{北原 鉄朗}, 
駒谷
      和範, 
尾形 哲也, 
奥乃 博: 
    ``OnomaTree:擬音語と木構造を併用した環境音検索インターフェース'', 
    {\it 情報処理学会 第69回全国大会,} 3N-7, March 2007. 

\item 
\underline{北原 鉄朗}, 
橋田 光代, 
片寄 晴弘: 
    ``音楽情報処理のための共通データフォーマットCrestMuseXML−全体構想と基本設計方針−'', 
    {\it 日本音響学会2007年秋季研究発表会 講演論文集,} 2-1-4, September 2007. 

\item 
\underline{北原 鉄朗}, 
後藤
      真孝, 
奥乃 博, 
片寄 晴弘: 
    ``Instrogram:多重奏中の楽器構成に関する確率論的表現法'', 
    {\it 電子情報通信学会2008年総合大会,
    } AS-5-4, March 2008. 

\item 
風谷 真志, 
\underline{北原 鉄朗}, 
片寄 晴弘: 
    ``確率文脈自由文法を用いた事例参照型自動作曲システム'', 
    {\it 情報処理学会 第70回全国大会,} 3X-3, March 2008. 

\item 
小林 一樹, 
\underline{北原 鉄朗}: 
    ``効率的なロボットプログラミング環境の実現に向けて'', 
    {\it 第22回人工知能学会全国大会,} 2G1-1, May 2008. 

\item 
\underline{北原 鉄朗}, 
片寄 晴弘: 
    ``MIDIデータのベロシティを異なる音源に適応させる試み'', 
    {\it 日本音響学会2008年秋季研究発表会 講演論文集,} 1-9-16, September 2008. 

\item 
山川 暢英, 
\underline{北原 鉄朗}, 
高橋 徹, 
駒谷 和範, 
尾形 哲也, 
奥乃 博: 
    ``環境音から擬音語への自動変換における特徴量抽出法の検討'', 
    {\it 情報処理学会第72回全国大会,
    } 3U-9, March 2010. 

\item 
水本 直希, 
\underline{北原 鉄朗}, 
片寄 晴弘: 
    ``エレキギターにおける演奏情報の特徴抽出'', 
    {\it 情報処理学会第72回全国大会,
    } 5T-1, March 2010. 

\item 
村主 大輔, 
森勢 将雅, 
\underline{北原 鉄朗}, 
片寄 晴弘: 
    ``奄美大島民謡風歌声合成のためのコブシに着目した歌声の特徴分析'', 
    {\it 情報処理学会第72回全国大会,
    } 6U-4, March 2010. 

\end{Enumerate}

\section*{口頭発表}
\begin{Enumerate}
  
\item 
\underline{北原 鉄朗}, 
後藤
      真孝, 
奥乃 博: 
    ``音高による音色変化と未知楽器の問題を考慮した楽器音の音源同定'', 
    {\it 日本音響学会関西支部 第6回若手研究者交流研究発表会,
    } December 2003. 

\item 
\underline{北原 鉄朗}, 
後藤
      真孝, 
駒谷
      和範, 
尾形 哲也, 
奥乃 博: 
    ``多重奏の音源同定における音の重なりに対する頑健性の改善'', 
    {\it 日本音響学会関西支部 第8回若手研究者交流研究発表会,
    } December 2005. 
{\bf (若手奨励賞受賞)}
\end{Enumerate}

\section*{解説記事}
\begin{Enumerate}
  
\item 
奥乃 博, 
\underline{北原 鉄朗}, 
吉井 和佳: 
    ``楽曲の特徴量抽出と検索技術'', 
    {\it 電気学会誌,
    } 特集「音響機器は進歩している」, Vol.127, No.7, pp.417--420, July 2007. 

\item 
\underline{北原 鉄朗}: 
    ``音楽情報処理最前線! 楽器で音楽が探せる 「楽器認識技術」が叶える音楽の新しい聴き方・探し方'', 
    {\it DTM Magazine,
    } Vol.176, pp.102--103, February 2009. 

\item 
平井 重行, 
橋田 光代, 
\underline{北原 鉄朗}, 
竹川 佳成, 
片寄 晴弘: 
    ``音楽とヒューマンインタフェース'', 
    {\it 情報処理,
    } 特集「音楽処理技術の最前線」, Vol.50, No.8, pp.756--763, August 2009. 

\item 
\underline{北原 鉄朗}: 
    ``私のブックマーク「音楽情報処理」'', 
    {\it 人工知能学会誌,
    } Vol.24, No.5, pp.921--929, November 2009. 

\end{Enumerate}

    \section*{章分担}
    \begin{Enumerate}
    
\item 
\underline{Tetsuro Kitahara}: 
    ``Mid-level Representations of Musical Audio Signals for Music Information Retrieval'', 
    {\it Advances in Music Information Retrieval,
    } Studies in Computational Intelligence 274, (Zbigniew W. Ras and Alicja A. Wieczorkowska (Eds.)), Springer, February 2010. 

    \end{Enumerate}
  
\section*{翻訳}
\begin{Enumerate}
  
\item 
Francois Pachet(著), 
北原 鉄朗(訳): 
    ``デジタル音楽配信のためのコンテンツ管理'', 
    {\it Communications of the ACM 日本語版,
    } Vol.4, No.2, pp.1--6, June 2004. 

\item 
Bryan Pardo(著), 
北原 鉄朗(訳): 
    ``音楽情報検索'', 
    {\it Communications of the ACM 日本語版,
    } Vol.6, No.2, pp.1--3, 2007. 

\item 
Avery Wang(著), 
北原 鉄朗(訳): 
    ``Shazam 音楽認識サービス'', 
    {\it Communications of the ACM 日本語版,
    } Vol.6, No.2, pp.17--21, 2007. 

\end{Enumerate}

   \section*{招待講演・パネルディスカッションなど}
   \begin{itemize}
   
\item 

    ``パネルディスカッション「音楽情報処理研究者\{に,が\}望むこと」'', 
    {\it 情報処理学会第51回音楽情報科学研究会,
    } パネリスト, August 2003. 

\item 

    ``新博士によるパネルディスカッション1「博士への道のりと将来への夢」'', 
    {\it 情報処理学会第71回音楽情報科学研究会,
    } パネリスト, August 2007. 

\item 

    ``パネルディスカッション「``音''研究の未来」'', 
    {\it 情報処理学会 音楽情報科学研究会・音声言語情報処理研究会 特別合同企画,
    } パネリスト, February 2008. 

\item 

    ``音楽の信号処理とパターン処理の基礎技術:入門と実践'', 
    {\it 情報処理学会 第76回音楽情報科学研究会 チュートリアル,
    } 講師, August 2008. 

\item 

    ``パネルディスカッション:作るだけでいいの?調べるだけでいいの?'', 
    {\it 情報処理学会第78回音楽情報科学研究会・日本音響学会音楽音響研究会 合同特別企画,
    } パネリスト, December 2008. 

\item 

    ``CrestMuseXML Toolkitで始める音楽情報処理入門'', 
    {\it 情報処理学会 第80回音楽情報科学研究会 チュートリアル,
    } 講師, May 2009. 

   \end{itemize}
 
\section*{特許}
\begin{itemize}
  
\item 
鍵盤楽器支援装置及び鍵盤楽器支援システム,特開2006-145681号(2006年6月8日),特願2004-333279(2004年11月17日),発明者:武田 正之,石田
      克久,北原 鉄朗.\par

\item 
楽器音認識方法,楽器アノテーション方法,及び楽曲検索方法,特願2006-058649号(2006年3月3日),特開2007-240552号(2007年9月20日),発明者:北原
      鉄朗,奥乃 博. \par

\end{itemize}

\section*{助成金}
\begin{itemize}
  
\item 
(財)C\&C振興財団 国際会議論文発表者助成 採択(IEA/AIE-2003での発表に対して)\par

\item 
(財)情報科学国際交流財団 研究者海外派遣助成 採択(ICME 2003での発表に対して)\par

\item 
(財)原総合知的通信システム基金 国際会議論文発表助成 採択(ICASSP 2004での発表に対して)\par

\item 
(財)電気通信普及財団 海外渡航旅費援助 採択(ISMIR 2005での発表に対して)\par

\item 
(財)立石科学技術振興財団 国際交流助成 採択(ICASSP 2006での発表に対して)\par

\item 
(財)電気通信普及財団 海外渡航旅費援助 採択(ISM 2008での発表に対して)\par

\item 
平成15年度 ASTEM学生ベンチャー奨励金制度 奨励金採択\par
「即興演奏の不自然な旋律を自動的に補正する機能を組み込んだ電子楽器の開発」\par

\item 
平成16年度 SCAT研究奨励金 採択\par
「音楽音響信号に対するMPEG-7タグの自動付与および音楽情報検索への応用」\par

\item 
21世紀COE「知識社会基盤構築のための情報学拠点形成」平成16年度 若手リーダーシップ養成プログラム研究費 採択\par
「高度な音楽検索実現のための音楽音響信号に対するMPEG-7タグの自動付与」\par

\item 
日本学術振興会 科学研究費補助金 特別研究員研究奨励費(平成17\&\#65374;18年度)\par
「音楽のディジタルアーカイブ化のためのMPEG-7タグの設計と自動付与」\par

\item 
日本学生支援機構 第1種奨学金「特に優れた業績による返還免除」認定(全額)\par

\end{itemize}

\section*{受賞}
\begin{itemize}
  
  \item 
  情報処理学会 第64回全国大会学生奨励賞
  \item 
  電気通信普及財団 第19回テレコムシステム技術学生賞 受賞
  \item 
  情報処理学会 第66回全国大会学生奨励賞
  \item 
  情報処理学会 第67回全国大会大会奨励賞
  \item 
  IEEE関西支部 第3回学生研究奨励賞 受賞
  \item 
  第3回IPSJ Digital Courier船井若手奨励賞
  \item 
  第2回京都大学総長賞受賞
  \item 
  日本音響学会関西支部 第8回若手研究者交流研究発表会若手奨励賞受賞
\end{itemize}

\section*{学会活動}
\begin{itemize}
  
\item 
情報処理学会 音楽情報科学研究会 主査(2015年度〜2016年度)\par

\item 
2016年度人工知能学会全国大会(第30回) 大会委員\par

\item 
Special Session on Music Information Processing, the IEEE 8th International Conference on Knowledge and Systems Engineering (KSE 2016), Session Organier\par

\item 
情報処理学会論文誌「エンターテインメントコンピューティング」特集(2016年12月発行予定)編集委員会 編集委員\par

\item 
情報処理学会誌「音楽を軸に拡がる情報科学」特集(2016年6月号)ゲストエディタ\par

\item 
情報処理学会論文誌「音楽情報処理技術の進歩とその拡がり」特集(2016年5月号)編集委員会 幹事\par

\item 
2015年度人工知能学会全国大会(第29回) 大会委員\par

\item 
情報処理学会 第76回全国大会 プログラム編成WG 委員\par

\item 
情報処理学会/電子情報通信学会 第13回情報科学技術フォーラム(FIT 2014) プログラム委員会 委員\par

\item 
Special Session on Hot Topics in Music Information Processing, the 12th IEEE International Conference on Signal Processing, Session Co-organizer\par

\item 
日本音響学会2014年春季研究発表会 実行委員\par

\item 
情報処理学会/電子情報通信学会 第12回情報科学技術フォーラム(FIT 2013) 研究会担当委員\par

\item 
情報処理学会論文誌「音楽情報処理の新展開(音楽情報科学研究会20周年記念特集)」特集(2013年4月号)編集委員会 編集委員\par

\item 
情報処理学会 音楽情報科学研究会 幹事(2011年度〜2014年度)\par

\item 
電子情報通信学会 和文論文誌D 編集委員会 編集委員(2011年度〜2014年度)\par

\item 
情報処理学会 音楽情報科学研究会 運営委員(2007年度\&\#65374;2010年度)\par

\item 
ISMIR 2009, Local Organizing Committee Chair\par

\item 
科学技術新興機構デジタルメディア領域主催シンポジウム「表現の未来へ」推進委員(2007年度)\par

\item 
ピアノ演奏表情付けコンテスト「Rencon」Committee Member (2007年度〜2011年度)\par

\item 
論文誌査読(複数回):情報処理学会論文誌,電子情報通信学会論文誌,日本音響学会誌, 人工知能学会論文誌, IEEE Transactions on Acoustics, Speech, and Language, IEEE Journal of Selected Topics in Signal Processing, 芸術科学会, ヒューマンインタフェース学会\par

\item 
論文誌査読(1回のみ):Journal of New Music Research, Signal Processing,日本神経回路学会誌\par

\item 
国際会議論文査読:ISMIR 2007, WASPAA 2007, ISMIR 2008, ISMIR 2009, ISMIR 2010, SAPA 2010, ISMIR 2011, ACE 2015, ACE 2016, IEEE-KSE 2016\par

\item 
国内会議論文査読:FIT 2008, FIT 2009\par

\end{itemize}

\end{document}
