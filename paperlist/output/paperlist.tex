
\documentstyle[a4j,11pt]{jarticle}

\newcounter{savedenumi}
\newenvironment{Enumerate}
{\begin{enumerate} \setcounter{enumi}{\thesavedenumi} 
\setlength{\itemsep}{7pt} }
{\setcounter{savedenumi}{\theenumi} \end{enumerate}}

\title{研究業績
  
}
\author{北原 鉄朗}
\date{}

\begin{document}
\maketitle


\section*{学位論文}
\begin{Enumerate}
  
\item 
\underline{北原 鉄朗}: 
    ``Computational Musical Instrument Recognition and Its Application to Content-based Music
      Information Retrieval'', 
    博士論文 京都大学大学院情報学研究科, February 2007. 
{\bf (第2回京都大学総長賞受賞)}
\end{Enumerate}

\section*{学術論文}
\begin{Enumerate}
  
\item 
\underline{北原 鉄朗}, 
後藤 真孝, 
奥乃 博: 
    ``音高による音色変化に着目した楽器音の音源同定:F0依存多次元正規分布に基づく識別手法'', 
    {\it 情報処理学会論文誌,
    } Vol.44, No.10, pp.2448--2458, October 2003. 
{\bf (電気通信普及財団 第19回テレコムシステム技術学生賞 受賞)}
\item 
\underline{北原 鉄朗}, 
後藤 真孝, 
奥乃 博: 
    ``音響的類似性を反映した楽器の階層表現の獲得とそれに基づく未知楽器のカテゴリーレベルの音源同定'', 
    {\it 情報処理学会論文誌,
    } 特集「音楽情報科学」, Vol.45, No.3, pp.680--689, March 2004. 

\item 
石田 克久, 
\underline{北原 鉄朗}, 
武田 正之: 
    ``N-gramによる旋律の音楽的適否判定に基づいた即興演奏支援システム'', 
    {\it 情報処理学会論文誌,
    } 特集「インタラクション:技術と展開」, Vol.46, No.7, pp.1549--1559, July 2005. 

\item 
\underline{Tetsuro Kitahara}, 
Masataka Goto, 
and 
Hiroshi
      G. Okuno: 
    ``Pitch-dependent Identification of Musical Instrument Sounds'', 
    {\it Applied Intelligence,
    } Vol.23, No.3, pp.267--275, December 2005. 

\item 
藤原
      弘将, 
\underline{北原 鉄朗}, 
後藤
      真孝, 
駒谷
      和範, 
尾形 哲也, 
奥乃 博: 
    ``伴奏音抑制と高信頼度フレーム選択に基づく楽曲の歌手名同定手法'', 
    {\it 情報処理学会論文誌,
    } 特集「情報処理技術のフロンティア」, Vol.47, No.6, pp.1831--1843, July 2006. 

\item 
\underline{北原 鉄朗}, 
後藤
      真孝, 
駒谷
      和範, 
尾形 哲也, 
奥乃 博: 
    ``多重奏を対象とした音源同定: 混合音テンプレートを用いた音の重なりに頑健な特徴量の重みづけ および音楽的文脈の利用'', 
    {\it 電子情報通信学会論文誌,
    } Vol.J89-D, No.12, pp.2721--2733, December 2006. 

\item 
\underline{Tetsuro Kitahara}, 
Masataka Goto, 
Kazunori Komatani, 
Tetsuya
      Ogata, 
and 
Hiroshi
      G. Okuno: 
    ``Instrument Identification in Polyphonic Music: Feature Weighting to Minimize Influence of
      Sound Overlaps'', 
    {\it EURASIP Journal on Advances in Signal Processing,
    } Special Issue on Music Information Retrieval based on Signal Processing, Vol.2007, No.51979, pp.1--15, 2007. 

\item 
\underline{Tetsuro Kitahara}, 
Masataka Goto, 
Kazunori Komatani, 
Tetsuya
      Ogata, 
and 
Hiroshi
      G. Okuno: 
    ``Instrogram: Probabilistic Representation of Instrument Existence for Polyphonic Music'', 
    {\it IPSJ Journal,
    } Special Issue on Convenient, Familiar Music Information Processing, Vol.48, No.1, pp.214--226, January 2007. 
{\bf (第3回IPSJ Digital Courier船井若手奨励賞)}(also published in IPSJ Digital Courier Vol.3, No.1, pp.1--13)
\item 
\underline{北原 鉄朗}, 
勝占 真規子, 
片寄 晴弘, 
長田 典子: 
    ``ベイジアンネットワークを用いた自動コードヴォイシングシステム'', 
    {\it 情報処理学会論文誌,
    } 特集「音楽情報処理」, Vol.50, No.3, pp.1067--1078, March 2009. 

\item 
橋田 光代, 
松井 淑恵, 
\underline{北原 鉄朗}, 
片寄 晴弘: 
    ``ピアノ名演奏の演奏表現情報と音楽構造情報を対象とした音楽演奏表情データベースCrestMusePEDBの構築'', 
    {\it 情報処理学会論文誌,
    } 特集「音楽情報処理」, Vol.50, No.3, pp.1090--1099, March 2009. 

\item 
Hiromasa Fujihara, 
Masataka Goto, 
\underline{Tetsuro Kitahara}, 
and 
Hiroshi
      G. Okuno: 
    ``Singing Voice Representation Robust to Accompaniment Sounds and Its Application to Singer
      Identification and Vocal-timbre-similarity-based Music Information Retrieval'', 
    {\it IEEE Transaction on Audio, Speech, and Language Processing,
    } Special Issue on Signal Models and Representation of Musical and Environmental Sounds, Vol.18, No.3, pp.638--648, March 2010. 

\item 
松原 正樹, 
深山 覚, 
奥村 健太, 
寺村 佳子, 
大村 英史, 
橋田 光代, 
\underline{北原 鉄朗}: 
    ``創作過程の分類に基づく自動音楽生成研究のサーベイ'', 
    {\it コンピュータソフトウェア(日本ソフトウェア科学会 学会誌),
    } Vol.30, No.1, pp.101--118, March 2013. 

\item 
Syunpei Suzuki, 
and 
\underline{Tetsuro Kitahara}: 
    ``Four-part Harmonization Using Bayesian Networks: Pros and Cons of Introducing Chord Nodes'', 
    {\it Journal of New Music Research,
    } Vol.43, No.3, pp.331--353, September 2014. 

\item 
栗原 拓也, 
木下 尚洋, 
山口 竜之介, 
横溝 有希子, 
竹腰 美夏, 
馬場 哲晃, 
\underline{北原 鉄朗}: 
    ``カラオケを盛り上げるためのタンバリン演奏支援システム'', 
    {\it 情報処理学会論文誌,
    } Vol.58, No.5, pp.1073--1092, May 2017. 

\item 
Shugo Ichinose, 
Souta Mizuno, 
Shun Shiramatsu, 
and 
\underline{Tetsuro Kitahara}: 
    ``Two Approaches to Supporting Improvisational Ensemble
for Music Beginners based on Body Motion Tracking'', 
    {\it International Journal of Smart Computing and Artificial Intelligence,
    } Vol.3, No.1, pp.55--70, 2019. 

\item 
栗原 一貴, 
植村 あい子, 
板谷 あかり, 
\underline{北原 鉄朗}, 
長尾 確: 
    ``Picognizer: 電子音の認識のための JavaScript ライブラリの開発と評価'', 
    {\it 情報処理学会論文誌,
    } Vol.60, No.2, pp.397--410, February 2019. 

\item 
Mina Shiraishi, 
Kozue Ogasawara, 
and 
\underline{Tetsuro Kitahara}: 
    ``HamoKara: A System that Enables Amateur Singers to Practice Backing Vocals for Karaoke'', 
    {\it Journal of Information Processing,
    } Vol.27, pp.683--692, November 2019. 

\item 
Yusuke Tsuchiya, 
and 
\underline{Tetsuro Kitahara}: 
    ``A Non-notewise Melody Editing Method for Supporting Musically Untrained People's Music Composition'', 
    {\it Journal of Creative Music Systems,
    } Vol.3, No.1, November 2019. 

\end{Enumerate}

\section*{ショートペーパー}
\begin{Enumerate}
  
\item 
石田 克久, 
\underline{北原 鉄朗}, 
武田 正之: 
    ``N-gramによる即興演奏の旋律補正'', 
    {\it 情報処理学会論文誌(テクニカルノート),
    } 特集「音楽情報科学」, Vol.45, No.3, pp.743--746, March 2004. 

\item 
\underline{北原 鉄朗}, 
戸谷 直之, 
徳網 亮輔, 
片寄 晴弘: 
    ``BayesianBand:ユーザとシステムが相互に予測し合うジャムセッションシステム'', 
    {\it 情報処理学会論文誌(テクニカルノート),
    } 特集「エンターテインメントコンピューティング」, Vol.50, No.12, pp.2949--2953, December 2010. 

\item 
土屋 裕一, 
\underline{北原 鉄朗}: 
    ``音符を単位としない旋律編集のための旋律概形抽出手法'', 
    {\it 情報処理学会論文誌(テクニカルノート),
    } Vol.54, No.4, pp.1302--1307, April 2013. 

\item 
\underline{Tetsuro Kitahara}, 
Shunsuke Hokari, 
and 
Tatsuya Nagayasu: 
    ``Supporting Jogging at an Even Pace by Synchronizing Music
Playback Speed with Runner's Pace'', 
    {\it IEICE Transactions on Information and Systems (Letter),
    } Vol.E98-D, No.4, pp.968--971, April 2015. 

\item 
鈴木 潤一, 
\underline{北原 鉄朗}: 
    ``複数人が同一空間で音楽を聴くための選曲・再生システム'', 
    {\it 情報処理学会論文誌(テクニカルノート),
    } Vol.57, No.12, pp.2526--2530, December 2016. 

\item 
草野 有沙, 
西 由佳梨, 
\underline{北原 鉄朗}: 
    ``ゲーム風演出で読書を促進するモバイルアプリケーション'', 
    {\it 情報処理学会論文誌(テクニカルノート),
    } Vol.60, No.11, pp.1978--1982, November 2019. 

\end{Enumerate}

\section*{国際会議}
\begin{Enumerate}
  
\item 
Misato Watanabe, 
Yosuke Onoue, 
Aiko Uemura, 
and 
\underline{Tetsuro Kitahara}: 
    ``A Web-based Application that Enables Users to Practice Wind Instrument Performance'', 
    {\it 15th International Symposium on CMMR, Online, Nov. 15-19, 2021,
    } . 

\item 
\underline{Tetsuro Kitahara}, 
Masataka Goto, 
and 
Hiroshi
      G. Okuno: 
    ``Musical Instrument Identification based on F0-dependent Multivariate Normal Distribution'', 
    {\it Proceedings of
      the 2003 IEEE International Conference on Acoustics, Speech, and Signal Processing
          (ICASSP 2003),
        } Vol.V, pp.421--424, April 2003. 
(Cancelled because of SARS)
\item 
\underline{Tetsuro Kitahara}, 
Masataka Goto, 
and 
Hiroshi
      G. Okuno: 
    ``Pitch-dependent Musical Instrument Identification and Its Application to Musical Sound
      Ontology'', 
    {\it Developments in Applied Artificial Intelligence --- Proceedings of the 16th
      International Conference on Industrial Engineering Applications of Artificial Intelligence and
      Expert Systems (IEA/AIE-2003),
    } LNAI 2718, (P. W. H. Chung, C. Hinde and M. Ali (Eds.)), pp.112--122, Springer, July 2003. 

\item 
\underline{Tetsuro Kitahara}, 
Masataka Goto, 
and 
Hiroshi
      G. Okuno: 
    ``Musical Instrument Identification based on F0-dependent Multivariate Normal Distribution'', 
    {\it Proceedings of the 2003 IEEE International Conference
      on Multimedia \& Expo
          (ICME 2003),
        } Vol.III, pp.409--412, July 2003. 
(Reprint of the paper published in ICASSP 2003)
\item 
\underline{Tetsuro Kitahara}, 
Masataka Goto, 
and 
Hiroshi
      G. Okuno: 
    ``Acoustical-similarity-based Musical Instrument Hierarchy and Its Application to Musical
      Instrument Identification'', 
    {\it Proceedings of the 2004 International Symposium on
      Musical Acoustics
          (ISMA 2004),
        } 3-S2-12, pp.397--300, April 2004. 
(abstract reviewed)
\item 
\underline{Tetsuro Kitahara}, 
Masataka Goto, 
and 
Hiroshi
      G. Okuno: 
    ``Category-level Identification of Non-registered Musical Instrument Sounds'', 
    {\it Proceedings of
      the 2004 IEEE International Conference on Acoustics, Speech, and Signal Processing
          (ICASSP 2004),
        } Vol.IV, pp.253--256, May 2004. 

\item 
Yohei Sakuraba, 
\underline{Tetsuro Kitahara}, 
and 
Hiroshi
      G. Okuno: 
    ``Comparing Features for Forming Music Streams in Automatic Music Transcription'', 
    {\it Proceedings of
      the 2004 IEEE International Conference on Acoustics, Speech, and Signal Processing
          (ICASSP 2004),
        } Vol.IV, pp.273--376, May 2004. 

\item 
Katsuhisa
      Ishida, 
\underline{Tetsuro Kitahara}, 
and 
Masayuki Takeda: 
    ``ism: Improvisation Supporting System based on Melody Correction'', 
    {\it Proceedings of the International Conference on New
      Interfaces for Musical Expression 
          (NIME 2004),
        } pp.177--180, June 2004. 

\item 
Takuya Yoshioka, 
\underline{Tetsuro Kitahara}, 
Kazunori Komatani, 
Tetsuya
      Ogata, 
and 
Hiroshi
      G. Okuno: 
    ``Automatic Chord Transcription with Concurrent Recognition of Chord Symbols and Boundaries'', 
    {\it Proceedings of
      the 5th International Conference on Music Information Retrieval
          (ISMIR 2004),
        } pp.100--105, October 2004. 

\item 
\underline{Tetsuro Kitahara}, 
Masataka Goto, 
Kazunori Komatani, 
Tetsuya
      Ogata, 
and 
Hiroshi
      G. Okuno: 
    ``Instrument Identification in Polyphonic Music: Feature Weighting with Mixed Sounds,
      Pitch-dependent Timbre Modeling, and Use of Musical Context'', 
    {\it Proceedings of
      the 6th International Conference on Music Information Retrieval 
          (ISMIR 2005),
        } pp.558--563, September 2005. 

\item 
Hiromasa Fujihara, 
\underline{Tetsuro Kitahara}, 
Masataka Goto, 
Kazunori Komatani, 
Tetsuya
      Ogata, 
and 
Hiroshi
      G. Okuno: 
    ``Singer Identification based on Accompaniment Sound Reduction and Reliable Frame Selection'', 
    {\it Proceedings of
      the 6th International Conference on Music Information Retrieval 
          (ISMIR 2005),
        } pp.329--336, September 2005. 

\item 
\underline{Tetsuro Kitahara}, 
Katsuhisa
      Ishida, 
and 
Masayuki Takeda: 
    ``ism: Improvisation Supporting Systems with Melody Correction and Key Vibration'', 
    {\it Entertainment Computing --- Proceedings of the 4th International Conference on
      Entertainment Computing (ICEC 2005),
    } LNCS 3711, (F. Kishino, Y. Kitamura, H. Kato and N. Nagata (Eds.)), pp.315--327, September 2005. 

\item 
\underline{Tetsuro Kitahara}, 
Masataka Goto, 
Kazunori Komatani, 
Tetsuya
      Ogata, 
and 
Hiroshi
      G. Okuno: 
    ``Instrogram: A New Musical Instrument Recognition Technique without Using Onset Detection
      nor F0 Estimation'', 
    {\it Proceedings of
      the 2006 IEEE International Conference on Acoustics, Speech, and Signal Processing
          (ICASSP 2006),
        } Vol.V, pp.229--232, May 2006. 
{\bf (IEEE関西支部 第3回学生研究奨励賞 受賞)}
\item 
Hiromasa Fujihara, 
\underline{Tetsuro Kitahara}, 
Masataka Goto, 
Kazunori Komatani, 
Tetsuya
      Ogata, 
and 
Hiroshi
      G. Okuno: 
    ``F0 Estimation Method for Singing Voice in Polyphonic Audio Signal based on Statistical
      Vocal Model and Viterbi Search'', 
    {\it Proceedings of
      the 2006 IEEE International Conference on Acoustics, Speech, and Signal Processing
          (ICASSP 2006),
        } Vol.V, pp.253--256, May 2006. 

\item 
Hiromasa Fujihara, 
\underline{Tetsuro Kitahara}, 
Masataka Goto, 
Kazunori Komatani, 
Tetsuya
      Ogata, 
and 
Hiroshi
      G. Okuno: 
    ``Speaker Identification under Noisy Environments by using Harmonic Structure Extraction
      and Reliable Frame Weighting'', 
    {\it Proceedings of the International Conference on
      Spoken Language Processing
          (ICSLP 2006),
        } September 2006. 

\item 
Katsutoshi Itoyama, 
\underline{Tetsuro Kitahara}, 
Kazunori Komatani, 
Tetsuya
      Ogata, 
and 
Hiroshi
      G. Okuno: 
    ``Automatic Feature Weighting in Automatic Transcription of Specified Part in Polyphonic
      Music'', 
    {\it Proceedings of
      the 7th International Conference on Music Information Retrieval 
          (ISMIR 2006),
        } October 2006. 

\item 
\underline{Tetsuro Kitahara}, 
Masataka Goto, 
Kazunori Komatani, 
Tetsuya
      Ogata, 
and 
Hiroshi
      G. Okuno: 
    ``Musical Instrument Recognizer ``Instrogram'' and Its Application to Music Retrieval based
      on Instrumentation Similarity'', 
    {\it Proceesings of the 8th IEEE International Symposium on
      Multimedia
          (ISM 2006),
        } pp.265--272, December 2006. 

\item 
\underline{Tetsuro Kitahara}, 
Makiko Katsura, 
Haruhiro Katayose, 
and 
Noriko Nagata: 
    ``Computational Model for Automatic Chord Voicing based on Bayesian Network'', 
    {\it Proceedings of the 10th International Conference on
      Music Perception and Cognition
          (ICMPC 2008),
        } pp.395--398, August 2008. 

\item 
\underline{Tetsuro Kitahara}, 
Masahiro Nishiyama, 
and 
Hiroshi
      G. Okuno: 
    ``Computational Model of Congruency between Music and Video'', 
    {\it Proceedings of the 10th International Conference on
      Music Perception and Cognition
          (ICMPC 2008),
        } August 2008. 
(abstract only)
\item 
Mitsuyo Hashida, 
Teresa M. Nakra, 
Haruhiro Katayose, 
Tadahiro Murao, 
Keiji Hirata, 
Kenji Suzuki, 
and 
\underline{Tetsuro Kitahara}: 
    ``Rencon: Performance Rendering Contest for Automated Music Systems'', 
    {\it Proceedings of the 10th International Conference on
      Music Perception and Cognition
          (ICMPC 2008),
        } pp.53--57, August 2008. 

\item 
Yusuke Tsuchihashi, 
\underline{Tetsuro Kitahara}, 
and 
Haruhiro Katayose: 
    ``Using Bass-line Features for Content-based MIR'', 
    {\it Proceedings of
      the 9th International Conference on Music Information Retrieval 
          (ISMIR 2008),
        } pp.620--625, September 2008. 

\item 
\underline{Tetsuro Kitahara}, 
Yusuke Tsuchihashi, 
and 
Haruhiro Katayose: 
    ``Music Genre Classification and Similarity Calculation Using Bass-line Features'', 
    {\it Proceesings of the 10th IEEE
      International Symposium on Multimedia, Workshop on Multimedia Audio and Speech Processing
          (ISM 2008 MASP Workshop),
        } pp.574--579, December 2008. 

\item 
\underline{Tetsuro Kitahara}, 
Naoyuki Totani, 
Ryosuke Tokuami, 
and 
Haruhiro Katayose: 
    ``BayesianBand: Jam Session System based on Mutual Prediction by User and System'', 
    {\it Entertainment Computing: Proceedings of the 10th International Conference on
    Entertainment Computing (ICEC 2009),
    } pp.179--184, September 2009. 

\item 
Nobuhide Yamakawa, 
\underline{Tetsuro Kitahara}, 
Toru Takahashi, 
Kozunori Komatani, 
Tetsuya Ogata, 
and 
Hiroshi G. Okuno: 
    ``Effects of Within- and Between-frame Temporal Variations in Power Spectra on Non-verbal Sound Recognition'', 
    {\it Proceedings of the 11th International Congress on Spoken Language Processing (Interspeech 2010),
    } September 2010. 

\item 
Nobuhide Yamakawa, 
Toru Takahashi, 
\underline{Tetsuro Kitahara}, 
Tetsuya Ogata, 
and 
Hiroshi G. Okuno: 
    ``Environmental Sound Recognition for Robot Audition using Matching-pursuit'', 
    {\it Modern Approaches in Applied Intelligence: 24th International Conference on Industrial Engineering and Other Applications of Applied Intelligent Systems, IEA/AIE 2011,
    } Lecture Notes in Artificial Intelligence 6704, (K.G. Mehrotra et al. (Eds.)), pp.1--10, June 2011. 

\item 
\underline{Tetsuro Kitahara}, 
Satoru Fukayama, 
Shigeki Sagayama, 
Haruhiro Katayose, 
and 
Noriko Nagata: 
    ``An Interactive Music Composition System based on Autonomous Maintenance of Musical Consistency'', 
    {\it Proceedings of the 8th Sound and Music Computing Conference,
    } pp.362--367, July 2011. 

\item 
Syunpei Suzuki, 
and 
\underline{Tetsuro Kitahara}: 
    ``Four-part Harmonization Using A Bayesian Network'', 
    {\it Proceedings of the 5th International Workshop on Machine Learning and Music (MML 2012),
    } in conjunction with ICML 2012, June 2012. 
(extended abstract)
\item 
Yuichi Tsuchiya, 
and 
\underline{Tetsuro Kitahara}: 
    ``Mutual Transform between Note Sequence and Melodic Envelope'', 
    {\it Proceedings of the 5th International Workshop on Machine Learning and Music (MML 2012),
    } in conjunction with ICML 2012, June 2012. 
(extended abstract)
\item 
\underline{Tetsuro Kitahara}, 
Syohei Kimura, 
Yuu Suzuki, 
and 
Tomofumi Suzuki: 
    ``Hummi-Com: Humming-based Music Composition System'', 
    {\it ACM Multimedia 2012 (Technical Demo),
    } pp.1321--1322, October 2012. 

\item 
Syunpei Suzuki, 
and 
\underline{Tetsuro Kitahara}: 
    ``Four-part Harmonization Using Probabilistic Models: Comparison of Models With and Without Chord Nodes'', 
    {\it Proceedings of the 10th Sound and Music Computing Conference (SMC 2013),
    } pp.628--633, August 2013. 

\item 
Yuichi Tsuchiya, 
and 
\underline{Tetsuro Kitahara}: 
    ``Melodic Outline Extraction Method for Non-note-level Melody Editing'', 
    {\it Proceedings of the 10th Sound and Music Computing Conference (SMC 2013),
    } pp.762--767, August 2013. 

\item 
Shogo Matsukata, 
Hiroko Terasawa, 
Masaki Matsubara, 
and 
\underline{Tetsuro Kitahara}: 
    ``Muscle Activity in Playing Trumpet: The Dependence on the Playable Pitch Region and the Experience of a Non-trumpet Brass Instrument Player'', 
    {\it Proceedings of the Stokholm Musical Acoustics Conference 2013 (SMAC 2013),
    } pp.529--533, August 2013. 

\item 
\underline{Tetsuro Kitahara}, 
Shunsuke Hokari, 
and 
Tatsuya Nagayasu: 
    ``Music Synchronizer with Runner's Pace
for Supporting Steady Pace Jogging'', 
    {\it HCI International 2014 - Posters’ Extended Abstracts,
    } Communications in Computer and Information Science, Vol.435, pp.343--348, Springer, June 2014. 
(abstract reviwed)
\item 
Kazuki Kogure, 
Masahiro Yoshinaga, 
Hikaru Suzuki, 
and 
\underline{Tetsuro Kitahara}: 
    ``A Spoken Dialogue System
for Noisy Environment'', 
    {\it HCI International 2014 - Posters’Extended Abstracts,
    } Communications in Computer and Information Science, Vol.435, pp.577--582, Springer, June 2014. 
(abstract reviewed)
\item 
\underline{Tetsuro Kitahara}, 
and 
Yuichi Tsuchiya: 
    ``Short-term and Long-term Evaluations of Melody Editing Method based on Melodic Outline'', 
    {\it Proceedings of the Joint Conference of the 40th International Computer Music Conference (ICMC 2014) and the 11th Sound and Music Computing Conference (SMC 2014),
    } pp.1204--1211, September 2014. 

\item 
\underline{Tetsuro Kitahara}, 
and 
Haruhiro Katayose: 
    ``CrestMuse Toolkit: A Java-based Framework for Signal and Symbolic Music Processing'', 
    {\it Proceedings of 12th IEEE International Conference on Signal Processing (ICSP 2014),
    } pp.616--620, October 2014. 

\item 
Masaki Otsuka, 
and 
\underline{Tetsuro Kitahara}: 
    ``An On-line Algorithm of Guitar Performance Transcription Using Non-negative Matrix Factorization'', 
    {\it Proceedings of 12th IEEE International Conference on Signal Processing (ICSP 2014),
    } pp.621--624, October 2014. 

\item 
Masaki Otsuka, 
and 
\underline{Tetsuro Kitahara}: 
    ``Towards Improvement of Transcription Accuracy of MIDI Guitar based on Integration with Audio Signal Processing'', 
    {\it The 15th Annual Meeting of the International Society for Music Information Retrieval (ISMIR 2014),
    } Late Breaking/Demo Session, October 2014. 
(not reviewed)
\item 
\underline{Tetsuro Kitahara}, 
Shogo Matsukata, 
Masaki Matsubara, 
and 
Hiroko Terasawa: 
    ``A Preliminary Experiment of Predicting Muscle Activity from Musical Acoustic Features'', 
    {\it Proceedings of the 7th International Workshop on Machine Learning and Music (MML 2014),
    } November 2014. 
(extended abstract)
\item 
\underline{Tetsuro Kitahara}, 
Kosuke Iijima, 
Misaki Okada, 
Yuji Yamashita, 
and 
Ayaka Tsuruoka: 
    ``A Loop Sequencer That Selects Music Loops based on the Degree of Excitement'', 
    {\it Proceedings of the 12th Sound and Music Computing Conference (SMC 2015),
    } pp.435--438, July 2015. 

\item 
Takuya Kurihara, 
Naohiro Kinoshita, 
Ryunosuke Yamaguchi, 
and 
\underline{Tetsuro Kitahara}: 
    ``A Tambourine Support System to Improve the Atmosphere of Karaoke'', 
    {\it Proceedings of the 12th Sound and Music Computing Conference (SMC 2015),
    } pp.515--520, July 2015. 

\item 
Masaki Otsuka, 
and 
\underline{Tetsuro Kitahara}: 
    ``Improving MIDI Guitar's Accuracy with NMF and Neural Net'', 
    {\it Proceedings of the 16th Internatiol Society for Music Information Retrieval Conference (ISMIR 2015),
    } pp.413--419, October 2015. 

\item 
Jun-ichi Suzuki, 
Naoyuki Suetsugu, 
and 
\underline{Tetsuro Kitahara}: 
    ``A Music Recommender for a Group of People'', 
    {\it The 2015 International Society of Music Information Retrieval (ISMIR 2015),
    } Late Breaking/Demo, October 2015. 
(not reviewed)
\item 
Tetsu Tanahashi, 
Yumie Takayashiki, 
and 
\underline{Tetsuro Kitahara}: 
    ``Support  System for Improving Speaking Skill in Job Interviews'', 
    {\it HCI International 2016 \&\#8211; Posters' Extended Abstracts, Part II,
    } Communication in Computer and Information Science (CCIS), Vol.618, pp.182--187, July 2016. 
(abstract reviewed)
\item 
Yuya Toyoda, 
Saori Nakajo, 
and 
\underline{Tetsuro Kitahara}: 
    ``An Android Application for Supporting Amateur Theatre'', 
    {\it HCI International 2016 \&\#8211; Posters' Extended Abstracts, Part II,
    } Communication in Computer and Information Science (CCIS), Vol.618, pp.558--563, July 2016. 
(abstract reviewed)
\item 
\underline{Tetsuro Kitahara}, 
and 
Masaki Matsubara: 
    ``Extracting Melodic Contour Using Wavelet-based Multi-resolution Analysis'', 
    {\it Proceedings of the 9th International Workshop on Music and Machine Learning (MML 2016),
    } in conjuction with ECML-PKDD 2016, pp.31--35, September 2016. 

\item 
Takuya Kurihara, 
Yukiko Yokomizo, 
Minatsu Takekoshi, 
Tetsuaki Baba, 
and 
\underline{Tetsuro Kitahara}: 
    `` A Tambourine Support System to Improve the Atmosphere of Karaoke: Support of Play by Multiple Players'', 
    {\it Proceedings of the 8th IEEE International Conference on Knowledge and Systems Engineering (KSE 2016),
    } pp.247--251, October 2016. 

\item 
Junichi Suzuki, 
and 
\underline{Tetsuro Kitahara}: 
    ``A Bluetooth-Networked Music Player for Playing Musical Pieces Stored in Separate Devices'', 
    {\it Proceedings of the 8th IEEE International Conference on Knowledge and Systems Engineering (KSE 2016),
    } October 2016. 

\item 
\underline{Tetsuro Kitahara}: 
    ``Smart Loop Sequencer: An Audio-based Approach for Ease of Music Creation'', 
    {\it 6th Joint Meething of the Acoustical Society of America (ASA) and the Acoustical Society of Japan (ASJ),
    } published in the Journal of the Acoustical Society of America (abstract only), Vol.140, November 2016. 
(invited talk)
\item 
\underline{Tetsuro Kitahara}, 
and 
Yuichi Tsuchiya: 
    ``A Machine Learning Approach to Support Music Creation by Musically Untrained People'', 
    {\it Proceedings of the Constructive Machine Learning Workshop,
    } in conjunction with NIPS 2016, December 2016. 

\item 
\underline{Tetsuro Kitahara}: 
    ``Towards Intuitive Music Creation Tools for Musically Untrained People'', 
    {\it Digital Music Research Network One-day Workshop 2016 (DMRN+11),
    } December 2016. 
(extended abstract)
\item 
Ryohei Ohno, 
Masanori Morise, 
\underline{Tetsuro Kitahara}: 
    ``The relationship between perception of cuteness and duration of voices'', 
    {\it Journal of Acoustic Society of America (the abstract for 5th Joint Meeting of the Acoustic Society of America and the Acoustic Society of Japan),
    } 5aSC44, Vol.140, No.4, pp.3399, December 2016. 
(abstract reviewed)
\item 
Tetsu Tanahashi, 
\underline{Tetsuro Kitahara}: 
    ``Relations on prosody of Japanese back-channeling word “hai”
and listener’s impression: An investigation with synthesized voices'', 
    {\it Journal of Acoustic Society of America (the abstract for 5th Joint Meeting of the Acoustic Society of America and the Acoustic Society of Japan),
    } 5aSC16, Vol.140, No.4, pp.3395, December 2016. 
(abstract reviewed)
\item 
\underline{Tetsuro Kitahara}, 
Sergio Giraldo, 
and 
Rafael Ram\&\#237;rez: 
    ``JamSketch: A Drawing-based Real-time Evolutionary Improvisation Support System'', 
    {\it Proceedings of the 2017 International Conference on New Interfaces for Musial Expression (NIME 2017),
    } pp.506--507, May 2017. 

\item 
Megumi Satou, 
\underline{Tetsuro Kitahara}, 
Hiroko Terasawa, 
and 
Masaki Matsubara: 
    ``Relationships between Abdominal and Around-Lip Muscle Activities
and Acoustic Features when Playing the Trumpet'', 
    {\it Proceedings of the 2017 International Symposium on Musical Acoustics,
    } pp.114--117, June 2017. 
{\bf (Best Student Paper Award)}
\item 
Shugo Ichinose, 
Souta Mizuno, 
Shun Shiramatsu, 
\underline{Tetsuro Kitahara}: 
    ``Improvisation Ensemble Support Systems for Music Beginners Based on Body Motion Tracking'', 
    {\it Proceedings of the 6th IIAI International Congress on Advanced Applied Informatics (IIAI AAI 2017),
    } pp.794-798, July 2017. 

\item 
\underline{Tetsuro Kitahara}, 
Jun Iwasaki, 
Haruka Koizumi, 
Keisuke Nagamura: 
    ``An Investigation of Pitch Perception of Poor-pitch Singers'', 
    {\it Proceedings of the 6th Conference of the Asia-Pacific Society for the Cognitive Science of Music (APSCOM 2017),
    } pp.51, August 2017. 
(extended abstract)
\item 
Yoshiki Matsuura, 
Tetsu Tanahashi, 
\underline{Tetsuro Kitahara}: 
    ``A Pattern Recognition Approach to Analyze Temporal Evolution of a Bassist’s Musical Styles'', 
    {\it Proceedings of the 2nd Conference on Computer Simulation of Musical Creativity,
    } September 2017. 

\item 
Ryohei Ohno, 
Masanori Morise, 
\underline{Tetsuro Kitahara}: 
    ``Relationship Between Perception of Cuteness in Female Voices and Their Durations'', 
    {\it Speech and Computer --- Proceedings of SPECOM 2017,
    } LNAI 10458, (Alexey Karpov, Rodmonga Potapova, and Iosif Mporas (Eds.)), pp.642--650, Springer, September 2017. 

\item 
\underline{Tetsuro Kitahara}: 
    ``Music Generation Using Bayesian Networks'', 
    {\it Machine Learning and Knowledge Discovery in Databases --- Proceedings of ECML PKDD 2017, Part III (Nectar Track),
    } LNAI 10536, (Michelangelo Ceci, Jaakko Hollm\&\#233;n, Ljup\&\#269;o Todorovski, Celine Vens, and Sa\&\#353;o D\&\#382;eroski (Eds.)), pp.368--372, Springer, September 2017. 

\item 
\underline{Tetsuro Kitahara}, 
Sergio Giraldo, 
Rafael Ramirez: 
    ``JamSketch: Improvisation Support System with GA-based Melody Creation from User’s Drawing'', 
    {\it Proceedings of the 13th International Symposium on Computer Music Multidisciplinary Research,
    } pp.352--363, September 2017. 

\item 
Tatsuro Yamada, 
\underline{Tetsuro Kitahara}, 
Hiroaki Arie, 
Tetsuya Ogata: 
    ``Four-part Harmonization: Comparison of a Bayesian Network and a Recurrent Neural Network'', 
    {\it Proceedings of the 13th International Symposium on Computer Music Multidisciplinary Research,
    } pp.137--148, September 2017. 

\item 
Jun'ichi Suzuki, 
\underline{Tetsuro Kitahara}: 
    ``A Music Player with Song Selection Function for a Group of People'', 
    {\it Proceedings of the 18th International Society for Music Information Retrieval Conference (ISMIR 2017),
    } pp.229--234, October 2017. 

\item 
Souta Mizuno, 
Shugo Ichinose, 
Shun Siramatsu, 
\underline{Tetsuro Kitahara}: 
    ``Support System of Improvisational Ensemble Based on User’s Motion Using Smartphone Sensors'', 
    {\it Proceedings of 12th International Conference on Knowledge, Information and Creativity Support System (KICSS 2017),
    } pp.143--148, November 2017. 

\item 
Kazutaka Kurihara, 
Akari Itaya, 
Aiko Uemura, 
\underline{Tetsuro Kitahara}, 
Katashi Nagao: 
    ``Picognizer: A JavaScript Library for Detecting and Recognizing Synthesized Sounds'', 
    {\it Advances in Computer Entertainmento Technology --- Proceedings of ACE 2017,
    } LNCS 10714, (Adrian David Cheok, Masahiko Inami, and Teresa Rom\&\#227;o (Eds.)), pp.339--359, Springer, December 2017. 

\item 
Mina Shiraishi, 
Kozue Ogasawara, 
and 
\underline{Tetsuro Kitahara}: 
    ``HamoKara: A System for Practice of Backing Vocals for Karaoke'', 
    {\it Proceedings of 15th Sound and Music Computing Conference (SMC 2018),
    } pp.511--518, July 2018. 

\item 
Aiko Uemura, 
and 
\underline{Tetsuro Kitahara}: 
    ``Preliminary Study on Morphing of Chord
Progression'', 
    {\it 
Proceedings of 3rd International Conference on Computer Simulation of Musical Creativity (CSMC 2018),
    } August 2018. 

\item 
Mai Udagawa, 
Aiko Uemura, 
and 
\underline{Tetsuro Kitahara}: 
    ``Support System for Excercising Guitar Chord Performance'', 
    {\it 3rd International Conference on Computer Simulation of Musical Creativity (CSMC 2018),
    } Late Breaking Abstracts, August 2018. 
(not reviewed)
\item 
\underline{Tetsuro Kitahara}, 
Yasuyuki Saito, 
Sergio Giraldo, 
and 
Rafael Ram\&\#237;rez: 
    ``An improvisation System for Disabilities based on Melody Creation with Gaze Control'', 
    {\it 3rd International Conference on Computer Simulation of Musical Creativity (CSMC 2018),
    } Late Breaking Abstracts, August 2018. 
(not reviewed)
\item 
Souta Mizuno, 
\underline{Tetsuro Kitahara}, 
Shun Shiramatsu, 
and 
Shugo Ichinose: 
    ``JamGesture: An Improvisation Support System Based on Physical Gesture Observed with Smartphone'', 
    {\it Proceedings of the 24th ACM Symposium on Virtual Reality Software and Technology (VRST 2018),
    } Poster Session, No.101, November 2018. 

\item 
\underline{Tetsuro Kitahara}, 
Sergio Giraldo, 
and 
Rafael Ram\&\#305;\&\#769;rez: 
    ``JamSketch: Improvisation Support
System with GA-Based Melody Creation
from User’s Drawing'', 
    {\it Music Technology with Swing --- 13th International Symposium, CMMR 2017, Matosinhos, Portugal, September 25\&\#8211;28, 2017, Revised Selected Papers,
    } LNCS 11265, (Mitsuko Aramaki
Matthew E. P. Davies
Richard Kronland-Martinet
S\&\#248;lvi Ystad (Eds.)), pp.509--521, Springer, December 2018. 

\item 
Tatsuro Yamada, 
\underline{Tetsuro Kitahara}, 
Hiroaki Arie, 
and 
Tetsuya Ogata: 
    ``Four-Part Harmonization: Comparison
of a Bayesian Network and a Recurrent
Neural Network'', 
    {\it Music Technology with Swing --- 13th International Symposium, CMMR 2017, Matosinhos, Portugal, September 25\&\#8211;28, 2017, Revised Selected Papers,
    } LNCS 11265, (Mitsuko Aramaki
Matthew E. P. Davies
Richard Kronland-Martinet
S\&\#248;lvi Ystad (Eds.)), pp.213--225, Springer, December 2018. 

\item 
Seiya Masuda, 
Eriko Aiba, 
and 
\underline{Tetsuro Kitahara}: 
    ``An Investigation towards Verbally Controllable Equalizer for Singing Voices'', 
    {\it Proceedings of the 5th Workshop on Intelligent Music Production (WIMP 2019),
    } September 2019. 

\item 
Ayumi Shiga, 
and 
\underline{Tetsuro Kitahara}: 
    ``Generating Walking Bass Lines with HMM'', 
    {\it Proceedings of the 14th International Symposium on Computer Music Multidisciplinary Research (CMMR 2019),
    } pp.83--90, October 2019. 

\item 
Aiko Uemura, 
and 
\underline{Tetsuro Kitahara}: 
    ``Morphing-Based Reharmonization using LSTM-VAE'', 
    {\it The 2020 Joint Conference on AI Music Creativity (CSMC + MuMe 2020),
    } October 2020. 

\item 
Mio Kusachi, 
Aiko Uemura, 
and 
\underline{Tetsuro Kitahara}: 
    ``A Piano Ballad Arrangement System'', 
    {\it The 2020 Joint Conference on AI Music Creativity (CSMC + MuMe 2020),
    } October 2020. 

\item 
Naoto Homma, 
Aiko Uemura, 
and 
\underline{Tetsuro Kitahara}: 
    ``Are Theme Songs Usable for Anime Retrieval?'', 
    {\it IEEE 4th International Conference on Multimedia Information Processing and Retrieval (IEEE MIPR 2021),
    } September 2021. 

\item 
Aiko Uemura, 
and 
\underline{Tetsuro Kitahara}: 
    ``Morphing-based Reharmonization with VAE: Reducing Dissonance with Consonance-based Loss Function'', 
    {\it Proceedings of the 3rd Conference on AI Music Creativity (AIMC 2022),
    } September 2022. 

\end{Enumerate}

\section*{国内査読付き会議}
\begin{Enumerate}
  
\item 
石田 克久, 
\underline{北原 鉄朗}, 
武田 正之: 
    ``ism:即興演奏の不自然な旋律を補正する演奏支援システム'', 
    {\it Proceedings of the 11th Workshop on Interactive
      Systems and Software
          (WISS 2003),
        } pp.19--24, December 2003. 

\item 
石田 克久, 
\underline{北原 鉄朗}, 
武田 正之: 
    ``演奏者に振動で情報提示する鍵盤楽器「ぶるぶるくん」'', 
    {\it Proceedings of the 12th Workshop on Interactive
      Systems and Software
          (WISS 2004),
        } pp.59--64, December 2004. 

\item 
三澤 由宇, 
細野 裕, 
仁科 章史, 
石田 克久, 
\underline{北原 鉄朗}, 
後藤 真孝, 
武田 正之: 
    ``Openism:旋律補正に基づく演奏支援機能付き遠隔地セッションシステム'', 
    {\it Proceedings of the 13th Workshop on Interactive
      Systems and Software
          (WISS 2005),
        } December 2005. 

\item 
戸谷 直之, 
\underline{北原 鉄朗}, 
片寄 晴弘: 
    ``楽器構成に着目した楽曲サムネイルとプレイリスト生成機能つき音楽プレイヤー'', 
    {\it インタラクション2008(インタラクティブ発表),
    } pp.173--174, March 2008. 

\item 
\underline{北原 鉄朗}, 
徳網 亮輔, 
戸谷 直之, 
橋本 寿政, 
片寄 晴弘: 
    ``BayesianBand:旋律の予測に基づいた自動伴奏システム'', 
    {\it インタラクション2009(インタラクティブ発表),
    } pp.31--32, March 2009. 

\item 
\underline{北原 鉄朗}, 
深山 覚, 
片寄 晴弘, 
嵯峨山 茂樹, 
長田 典子: 
    ``OrpheusBB:Human-in-the-loop型の自動作曲システム'', 
    {\it 情報処理学会 インタラクション2011(口頭発表),
    } pp.57--64, March 2011. 

\item 
栗原 一貴, 
板谷 あかり, 
植村 あい子, 
\underline{北原 鉄朗}: 
    ``Picognizer: 電子音の検出および認識のためのJavaScriptライブラリ'', 
    {\it 第21回インタラクティブシステムとソフトウェアに関するワークショップ論文集 (WISS 2017),
    } December 2017. 

\end{Enumerate}

\section*{国内研究会}
\begin{Enumerate}
  
\item 
\underline{北原 鉄朗}, 
後藤 真孝, 
奥乃 博: 
    ``音高による音色変化に着目した音源同定手法'', 
    {\it 情報処理学会 音楽情報科学 研究報告,} 2001-MUS-40-2, Vol.2001, No.45, pp.7--14, May 2001. 

\item 
\underline{北原 鉄朗}, 
後藤 真孝, 
奥乃 博: 
    ``楽器音を対象とした音源同定:音高による音色変化を考慮する識別手法の検討'', 
    {\it 情報処理学会 音楽情報科学 研究報告,} 2002-MUS-46-1, Vol.2002, No.63, pp.1--8, July 2002. 

\item 
\underline{北原 鉄朗}, 
後藤 真孝, 
奥乃 博: 
    ``音響的特徴に基づく楽器の階層表現の獲得とそれに基づくカテゴリーレベルの楽器音認識の検討'', 
    {\it 情報処理学会 音楽情報科学 研究報告,} 2003-MUS-51-9, Vol.2003, No.82, pp.51--58, August 2003. 

\item 
吉井 和佳, 
\underline{北原 鉄朗}, 
櫻庭 洋平, 
奥乃 博: 
    ``自己組織化マップによる教師なしクラスタリングを利用したドラム演奏の自動採譜'', 
    {\it 情報処理学会 音楽情報科学 研究報告,} 2003-MUS-51-8, Vol.2003, No.82, pp.43--40, August 2003. 

\item 
後藤 真孝, 
平田 圭二, 
片寄 晴弘, 
平井 重行, 
濱中 雅俊, 
武田 晴登, 
\underline{北原 鉄朗}: 
    ``パネルディスカッション「音楽情報処理研究者\{に,が\}望むこと」'', 
    {\it 情報処理学会 音楽情報科学 研究報告,} 2003-MUS-51-5, Vol.2003, No.82, pp.25--28, August 2003. 

\item 
石田 克久, 
\underline{北原 鉄朗}, 
武田 正之: 
    ``ism:即興演奏支援のためのリアルタイム旋律補正システム'', 
    {\it 情報処理学会 ヒューマンインターフェース研究会/音楽情報科学研究会 研究報告,
    } 2003-HI-106-2, 2003-MUS-52-2, Vol.2003, No.111, pp.9--15, November 2003. 

\item 
\underline{北原 鉄朗}, 
後藤 真孝, 
奥乃 博: 
    ``TimbreTree:音色の類似度に基づいた楽器の階層的分類'', 
    {\it 日本音響学会 音楽音響研究会 資料,
    } MA2004-7, Vol.23, No.2, pp.13--18, June 2004. 

\item 
吉岡 拓也, 
\underline{北原 鉄朗}, 
尾形 哲也, 
奥乃 博: 
    ``音楽音響信号を対象とした和音進行の認識'', 
    {\it 日本音響学会 音楽音響研究会 資料,
    } MA2004-8, Vol.23, No.2, pp.19--24, June 2004. 

\item 
\underline{北原 鉄朗}, 
後藤 真孝, 
奥乃 博: 
    ``混合音テンプレートを用いた多重奏の音源同定'', 
    {\it 情報処理学会 音楽情報科学 研究報告,} 2004-MUS-56-9, Vol.2004, No.84, pp.57--64, August 2004. 

\item 
吉岡 拓也, 
\underline{北原 鉄朗}, 
尾形 哲也, 
奥乃 博: 
    ``和音区間検出と和音名同定の相互依存性を解決する和音認識手法'', 
    {\it 情報処理学会 音楽情報科学 研究報告,} 2005-MUS-56-6, Vol.2004, No.84, pp.33--40, August 2004. 

\item 
浜中 雅俊, 
\underline{北原 鉄朗}, 
石田 克久, 
谷井 章夫, 
竹川 佳成, 
吉井 和佳, 
宮下 芳明, 
上 田 健太郎: 
    ``デモンストレーション:若手による研究紹介'', 
    {\it 情報処理学会 音楽情報科学 研究報告,} 2004-MUS-56-6, Vol.2004, No.84, pp.27--32, August 2004. 

\item 
\underline{北原 鉄朗}, 
石田 克久, 
武田 正之: 
    ``振動機能付鍵盤楽器「ぶるぶるくん」を用いた即興演奏支援システム'', 
    {\it 情報処理学会 音楽情報科学 研究報告,} 2005-MUS-60-5, Vol.2005, No.45, pp.25--30, May 2005. 

\item 
浜中 雅俊, 
李 昇姫, 
池月 雄哉, 
石原 一志, 
\underline{北原 鉄朗}, 
野池 賢二, 
中野 倫靖, 
梶 克彦, 
岡 良典, 
平田 圭二, 
松田 周, 
青木 忍, 
上田 健太郎: 
    ``デモンストレーション:若手による研究紹介II'', 
    {\it 情報処理学会 音楽情報科学 研究報告,} 2005-MUS-61-5, Vol.2005, No.82, pp.27--33, August 2005. 

\item 
藤原
      弘将, 
\underline{北原 鉄朗}, 
後藤
      真孝, 
駒谷
      和範, 
尾形 哲也, 
奥乃 博: 
    ``伴奏音抑制と高信頼度フレーム選択に基づく楽曲中の歌声の歌手名同定手法'', 
    {\it 情報処理学会 音楽情報科学 研究報告,} 2005-MUS-61-16, Vol.2005, No.82, pp.97--104, August 2005. 

\item 
\underline{北原 鉄朗}, 
後藤
      真孝, 
駒谷
      和範, 
尾形 哲也, 
奥乃 博: 
    ``Instrogram: 発音時刻検出とF0推定の不要な楽器音認識手法'', 
    {\it 情報処理学会 音楽情報科学 研究報告,} 2006-MUS-66, Vol.2006, No.90, pp.69--76, August 2006. 

\item 
浜中 雅俊, 
竹川 佳成, 
橋田 朋子, 
元川 洋一, 
馬場 哲晃, 
日暮 圭, 
 中野 倫靖, 
吉井 和佳, 
松原 正樹, 
梶 克彦, 
\underline{北原 鉄朗}: 
    ``デモンストレーション:若手による研究紹介III'', 
    {\it 情報処理学会 音楽情報科学 研究報告,} 2006-MUS-66-10, Vol.2006, No.90, pp.55--61, August 2006. 

\item 
浜中 雅俊, 
竹川 佳成, 
岩井 憲一, 
高橋 直也, 
 中野 倫靖, 
大石 康智, 
糸山 克寿, 
\underline{北原 鉄朗}, 
吉井 和佳: 
    ``デモンストレーション:若手による研究紹介IV'', 
    {\it 情報処理学会 音楽情報科学 研究報告,} 2006-MUS-67-3, Vol.2006, No.113, pp.9--14, October 2006. 

\item 
西山 正紘, 
\underline{北原 鉄朗}, 
駒谷
      和範, 
尾形 哲也, 
奥乃 博: 
    ``マルチメディアコンテンツにおける音楽と映像の調和度計算モデル'', 
    {\it 情報処理学会 音楽情報科学 研究報告,} 2007-MUS-69, Vol.2007, No.15, pp.31--36, February 2007. 

\item 
安部 武宏, 
\underline{北原 鉄朗}, 
糸山 克寿, 
柳田 益造: 
    ``撥弦の物理モデルを用いた音響信号からのパラメータ推定'', 
    {\it 日本音響学会音楽音響研究会資料,
    } MA2006-91, pp.35--40, March 2007. 

\item 
\underline{北原 鉄朗}, 
橋田 光代, 
片寄 晴弘: 
    ``音楽情報科学研究のための共通データフォーマットの確立を目指して'', 
    {\it 情報処理学会 音楽情報科学 研究報告,} 2006-MUS-66-12, Vol.2007, No.81, pp.149--154, August 2007. 

\item 
平田 圭二, 
梶 克彦, 
亀岡 弘和, 
\underline{北原 鉄朗}, 
齋藤 毅, 
武田 晴登, 
橋田 光代: 
    ``新博士にょるパネルディスカッション1「博士への道のりと将来への夢」'', 
    {\it 情報処理学会 音楽情報科学 研究報告,} 2007-MUS-71-7, Vol.2007, No.81, pp.39--42, August 2007. 

\item 
\underline{北原 鉄朗}, 
後藤
      真孝, 
奥乃 博, 
片寄 晴弘: 
    ``楽器音認識技術を用いた音楽の可視化'', 
    {\it Proceedings of
      Entertainment Computing 2007
          (EC2007),
        } pp.145--148, October 2007. 

\item 
橋田 光代, 
松井 淑恵, 
\underline{北原 鉄朗}, 
酒造 祐介, 
片寄 晴弘: 
    ``音楽演奏表情データベースCrestMusePEDB ver1.0の公開について'', 
    {\it 情報処理学会 音楽情報科学 研究報告,} 2007-MUS-72-1, Vol.2007, No.102, pp.1--6, October 2007. 

\item 
土橋 佑亮, 
\underline{北原 鉄朗}, 
片寄 晴弘: 
    ``音響信号を対象としたベースラインからの音楽ジャンル解析'', 
    {\it 情報処理学会 音楽情報科学/音声言語情報処理 研究報告,
    } 2008-MUS-74-38, 2008-MUS-SLP-70-38, Vol.2008, No.12, pp.217--224, February 2008. 

\item 
勝占 真規子, 
\underline{北原 鉄朗}, 
片寄 晴弘, 
長田 典子: 
    ``ベイジアンネットワークを用いたコード・ヴォイシング推定システム'', 
    {\it 情報処理学会 音楽情報科学/音声言語情報処理 研究報告,
    } 2008-MUS-74-29, 2008-MUS-SLP-70-29, Vol.2008, No.12, pp.163--168, February 2008. 

\item 
藤田 徹, 
\underline{北原 鉄朗}, 
片寄 晴弘, 
長田 典子: 
    ``アーティストの個性を表す音楽的特徴に関する一考察'', 
    {\it 情報処理学会 音楽情報科学/音声言語情報処理 研究報告,
    } 2008-MUS-74-35, 2008-MUS-SLP-70-35, Vol.2008, No.12, pp.199--204, February 2008. 

\item 
後藤
      真孝, 
亀岡 弘和, 
\underline{北原 鉄朗}, 
平賀 譲, 
緒方 淳, 
戸田 智基: 
    ``パネルディスカッション「``音''研究の未来」'', 
    {\it 情報処理学会 音楽情報科学/音声言語情報処理 研究報告,
    } 2008-MUS-74-10, 2008-SLP-70-10, Vol.2008, No.12, pp.57--58, February 2008. 

\item 
\underline{北原 鉄朗}, 
小林 一樹, 
片寄 晴弘: 
    ``演奏家型人形を利用した見えない演奏者の可視化の試み'', 
    {\it インタラクション2008(ポスター発表),
    } March 2008. 

\item 
\underline{北原 鉄朗}, 
片寄 晴弘: 
    ``CrestMuseXML (CMX) Toolkit ver.0.40について'', 
    {\it 情報処理学会 音楽情報科学 研究報告,} 2008-MUS-75-17, Vol.2008 , No.50, pp.95--100, May 2008. 

\item 
\underline{北原 鉄朗}, 
平田 圭二, 
竹川 佳成, 
中野 倫靖, 
森勢 将雅, 
吉井 和佳: 
    ``新博士によるパネルディスカッションII「楽しくさせる音楽,楽しくさせる研究」'', 
    {\it 情報処理学会 音楽情報科学 研究報告,} 2008-MUS-76-1, Vol.2008 , No.78, pp.1--4, August 2008. 

\item 
橋本 祐輔, 
\underline{北原 鉄朗}, 
片寄 晴弘: 
    ``音楽音響信号を対象とした指揮演奏システム:フェルマータ時における打楽器音抑制とスケジューラの検討'', 
    {\it 情報処理学会 音楽情報科学 研究報告,} 2008-MUS-76-7, Vol.2008 , No.78, pp.33--37, August 2008. 

\item 
三浦 雅展, 
江村 伯夫, 
\underline{北原 鉄朗}, 
若槻 尚斗, 
藤島 琢哉, 
西口 磯春, 
平田 圭二, 
柳田 益造, 
後藤 真孝: 
    ``パネルディスカッション:作るだけでいいの?調べるだけでいいの?'', 
    {\it 情報処理学会 音楽情報科学 研究報告,} 2008-MUS-78-11, Vol.2008, No.78, pp.59--66, December 2008. 

\item 
橋田 光代, 
片寄 晴弘, 
平田 圭二, 
\underline{北原 鉄朗}, 
鈴木 健嗣: 
    ``演奏表情付けコンテストICMPC-Rencon開催報告'', 
    {\it 情報処理学会 音楽情報科学 研究報告,} 2008-MUS-78-12, Vol.2008, No.78, pp.67--72, December 2008. 

\item 
\underline{北原 鉄朗}: 
    ``CrestMuseXML Toolkitで始める音楽情報処理入門'', 
    {\it 情報処理学会 音楽情報科学 研究報告,
    } 2009-MUS-50-1, May 2009. 

\item 
戸谷 直之, 
\underline{北原 鉄朗}, 
片寄 晴弘: 
    ``予測型ジャムセッションシステムBayesianBandにおける可視化機能の導入'', 
    {\it エンターテインメントコンピューティング2009,
    } September 2009. 

\item 
橋田 光代, 
\underline{北原 鉄朗}, 
鈴木健嗣, 
平田 圭二, 
片寄 晴弘: 
    ``演奏表情付けコンテストEC-Rencon'', 
    {\it エンターテインメントコンピューティング2009,
    } September 2009. 

\item 
橋田 光代, 
\underline{北原 鉄朗}, 
鈴木 健嗣, 
片寄 晴弘, 
平田 圭二: 
    ``演奏表情付けコンテストEC-Rencon開催報告'', 
    {\it 情報処理学会 音楽情報科学 研究報告,
    } 2009-MUS-83, November 2009. 

\item 
橋田 光代, 
松井 淑恵, 
\underline{北原 鉄朗}, 
片寄 晴弘: 
    ``音楽演奏表情データベースCrestMusePEDB ver. 2.4の概要とフレーズ構造に基づく演奏データ収録状況'', 
    {\it 情報処理学会 音楽情報科学 研究報告,
    } 2010-MUS-85-6, May 2010. 

\item 
橋田 光代, 
\underline{北原 鉄朗}, 
鈴木 健嗣, 
片寄 晴弘, 
平田 圭二: 
    ``Rencon Workshop 2010: 演奏表情付けコンテスト'', 
    {\it 情報処理学会 音楽情報科学 研究報告,
    } 2010-MUS-86-14, July 2010. 

\item 
水本 直希, 
\underline{北原 鉄朗}, 
片寄 晴弘: 
    ``事例データに基づくエレキギターの表情付けシステム'', 
    {\it 情報処理学会 音楽情報科学 研究報告,
    } 2009-MUS-87, October 2010. 

\item 
橋田 光代, 
松井 淑恵, 
馬場 隆, 
\underline{北原 鉄朗}, 
片寄 晴弘: 
    ``音楽演奏表情データベースCrestMusePEDB 3.0: 収録演奏の公開とフレーズ構造記述について'', 
    {\it 情報処理学会 音楽情報科学 研究報告,
    } 2011-MUS-89, February 2011. 

\item 
橋田 光代, 
\underline{北原 鉄朗}, 
鈴木 健嗣, 
片寄 晴弘, 
平田 圭二: 
    ``演奏表情付けコンテストSMC-Rencon開催報告'', 
    {\it 情報処理学会 音楽情報科学 研究報告,
    } 2011-MUS-92-4, October 2011. 

\item 
水本 直希, 
馬場 隆, 
\underline{北原 鉄朗}, 
片寄 晴弘: 
    ``エレキギターの表情付け支援システム「Guitar-Case Maker」'', 
    {\it 情報処理学会 音楽情報科学・音声言語情報処理 研究報告,
    } 2012-MUS-94-30/2012-SLP-90-30, January 2012. 

\item 
\underline{北原 鉄朗}, 
江村 伯夫: 
    ``パネルディスカッション「その研究って音楽の必要あるの?」'', 
    {\it 情報処理学会 音楽情報科学 研究報告,
    } 2012-MUS-95-6, June 2012. 

\item 
土屋 裕一, 
\underline{北原 鉄朗}: 
    ``旋律包絡抽出に基づく直感的な旋律編集手法'', 
    {\it 情報処理学会 音楽情報科学 研究報告,
    } 2012-MUS-95-9, June 2012. 

\item 
鈴木 峻平, 
竹内 俊雄, 
佐藤 挂亮, 
\underline{北原 鉄朗}: 
    ``ベイジアンネットワークを用いた四声体和声付け'', 
    {\it 情報処理学会 音楽情報科学 研究報告,
    } 2012-MUS-95-8, June 2012. 

\item 
橋田 光代, 
松井 淑恵, 
\underline{北原 鉄朗}, 
片寄 晴弘: 
    ``定量的ピアノ演奏分析のための音楽演奏表情データベース'', 
    {\it 情報処理学会 音楽情報科学 研究報告,
    } 2013-MUS-99-54, May 2013. 

\item 
松方 翔吾, 
寺澤 洋子, 
松原 正樹, 
\underline{北原 鉄朗}: 
    ``トランペット演奏時の音高や強度の変化が口唇周囲の筋肉に及ぼす影響'', 
    {\it 情報処理学会 音楽情報科学 研究報告,
    } 2013-MUS-99-39, May 2013. 

\item 
鈴木 峻平, 
\underline{北原 鉄朗}: 
    ``ベイジアンネットワークを用いた四声体和声付け:音の前後関係を考慮したモデルを用いた検討'', 
    {\it 情報処理学会 音楽情報科学 研究報告,
    } 2013-MUS-99-9, May 2013. 

\item 
岡田 美咲, 
山下 雄史, 
\underline{北原 鉄朗}: 
    ``音素材の自動挿入機能を備えたループシーケンサ'', 
    {\it 情報処理学会 音楽情報科学 研究報告,
    } 2013-MUS-100-36, September 2013. 

\item 
山内 雅史, 
篠本 亮, 
西脇 絵里子, 
小野澤 理沙, 
\underline{北原 鉄朗}: 
    ``Kinectとワイヤレスマウスを併用したダンス学習支援システムの試作'', 
    {\it Entertainment Computing 2013 (EC 2013),
    } October 2013. 

\item 
\underline{北原 鉄朗}, 
小暮 計貴, 
吉永 眞宏, 
鈴木 光: 
    ``騒音下における声の張り上げ現象の計算機による実現に向けて'', 
    {\it 人工知能学会第39回AIチャレンジ研究会,
    } March 2014. 

\item 
大塚 匡紀, 
\underline{北原 鉄朗}: 
    ``MIDIギターの精度向上を目指した音響信号処理の検討'', 
    {\it 情報処理学会 音楽情報科学 研究報告,
    } 2014-MUS-103-15, May 2014. 

\item 
小暮 計貴, 
\underline{北原 鉄朗}: 
    ``周囲の雑音やユーザーの聞き返しに基づいて音量調節を行う音声対話システム'', 
    {\it 情報処理学会 音楽情報科学 研究報告,
    } 2014-MUS-103-28, May 2014. 

\item 
栗原 拓也, 
木下 尚洋, 
山口 竜之介, 
\underline{北原 鉄朗}: 
    ``「Wiiタンバリン」を用いたタンバリン演奏支援機能付きカラオケシステム'', 
    {\it エンタテインメントコンピューティングシンポジウム2015論文集,
    } pp.37--39, September 2015. 

\item 
鈴木 潤一, 
末次 尚之, 
\underline{北原 鉄朗}: 
    ``友人同士で好みの楽曲を聴かせ合うスマートフォン用ミュージックプレイヤー'', 
    {\it エンタテインメントコンピューティングシンポジウム2015論文集,
    } pp.186--189, September 2015. 

\item 
大野 涼平, 
\underline{北原 鉄朗}: 
    ``韻律変換実現のための一試行:高橋みなみ風の音声を小嶋陽菜風に変えてみた'', 
    {\it エンタテインメントコンピューティングシンポジウム2015論文集,
    } pp.483--486, September 2015. 

\item 
棚橋 徹, 
高屋敷 弓恵, 
\underline{北原 鉄朗}: 
    ``音声の韻律分析及び表情の特徴抽出による面接支援システム'', 
    {\it 情報処理学会 音楽情報科学 研究報告,
    } Vol.2016-MUS-111, No.30, pp.1--5, May 2016. 

\item 
大野 涼平, 
森勢 将雅, 
\underline{北原 鉄朗}: 
    ``音声における「かわいらしさ」の知覚と聴取時間の関係性の検討'', 
    {\it 情報処理学会 音楽情報科学 研究報告,
    } Vol.2016-MUS-111, No.50, pp.1--5, May 2016. 
{\bf (学生奨励賞受賞)}
\item 
栗原 拓也, 
横溝 有希子, 
竹腰 美夏, 
馬場 哲晃, 
\underline{北原 鉄朗}: 
    ``スマートタンバリン:音と光で場を盛り上げるカラオケ支援システム'', 
    {\it 情報処理学会 音楽情報科学 研究報告,
    } 2017-MUS-114-3, February 2017. 

\item 
鈴木 潤一, 
\underline{北原 鉄朗}: 
    ``複数ユーザー間での楽曲推薦を実現するミュージックプレイヤー:楽曲類似度の導入と有効性の検証'', 
    {\it 情報処理学会 音楽情報科学 研究報告,
    } 2017-MUS-114-27, February 2017. 

\item 
\underline{北原 鉄朗}, 
Sergio Giraldo, 
Rafael Ramirez: 
    ``曲線描画に基づく即興演奏支援システム'', 
    {\it 情報処理学会 インタラクション2017(インタラクティブ発表),
    } 3-405-57, March 2017. 

\item 
水野 創太, 
一ノ瀬 修吾, 
白松 俊, 
\underline{北原 鉄朗}: 
    ``スマートフォンセンサーを用いた即興合奏のための身体動作認識機構の試作'', 
    {\it 情報処理学会 インタラクション2017(インタラクティブ発表),
    } 3-410-69, March 2017. 

\item 
南條 浩輝, 
高道 慎之介, 
\underline{北原 鉄朗}, 
森勢 将雅: 
    ``外国語音声を好みの声質にかえる技術の検討 - 聞きつづけたくなる外国語教材をめざして -'', 
    {\it 情報処理学会 音楽情報科学 研究報告(音学シンポジウム2017),
    } 2017-MUS-115-60, pp.1--3, June 2017. 

\item 
石山 俊之, 
蓮井 星良, 
\underline{北原 鉄朗}: 
    ``HMDを用いたヴァーチャルなドラム演奏環境の試作'', 
    {\it エンターテインメントコンピューティングシンポジウム2017論文集,
    } pp.295--297, September 2017. 

\item 
栗原 一貴, 
板谷 あかり, 
植村 あい子, 
\underline{北原 鉄朗}: 
    ``電子音の認識のためのJavaScriptライブラリの開発'', 
    {\it エンタテインメントコンピューティングシンポジウム2017論文集,
    } pp.1--10, September 2017. 
(プレミアムペーパー)
\item 
水野 創太, 
白松 俊, 
\underline{北原 鉄朗}, 
一ノ瀬 修吾: 
    ``JamGesture:スマートフォンを用いた身体動作による即興演奏支援システム'', 
    {\it 情報処理学会 音楽情報科学 研究報告,
    } 2017-MUS-117-4, pp.1--4, November 2017. 

\item 
本間 直人, 
\underline{北原 鉄朗}: 
    ``アニメの主題歌による類似アニメ検索の検討'', 
    {\it  情報処理学会 音楽情報科学/音声言語情報処理 研究報告(音学シンポジウム2018),
    } 2018-MUS-119-42 / 2018-SLP-122-42, pp.1--2, June 2018. 

\item 
植村 あい子, 
\underline{北原 鉄朗}: 
    ``コード進行に関するモーフィングの初期検討'', 
    {\it  情報処理学会 音楽情報科学/音声言語情報処理 研究報告(音学シンポジウム2018),
    } 2018-MUS-119-20 / 2018-SLP-122-20, pp.1--5, June 2018. 

\item 
竹川 佳成, 
\underline{北原 鉄朗}: 
    ``音楽情報科学のスーパーヒーローたち! シリーズI'', 
    {\it  情報処理学会 音楽情報科学 研究報告,
    } 2018-MUS-120-17, pp.1--1, August 2018. 

\item 
石山 俊之, 
\underline{北原 鉄朗}: 
    ``HMDを用いたヴァーチャルなドラム演奏環境の試作:合奏相手を表すヴァーチャルキャラクターの導入'', 
    {\it エンタテインメントコンピューティングシンポジウム2018論文集,
    } pp.76--79, September 2018. 

\item 
\underline{北原 鉄朗}: 
    ``メロディ生成における生成単位に関する一調査'', 
    {\it 情報処理学会 音楽情報科学 研究報告,
    } 2018-MUS-121-27, pp.1-4, November 2018. 

\item 
阿部 賢人, 
\underline{北原 鉄朗}: 
    ``運転中の音楽に変化を与え眠気を気付かせるシステム'', 
    {\it 情報処理学会 インタラクション2019(インタラクティブ発表),
    } 1B-29, March 2019. 

\item 
安原 茜, 
藤井 潤子, 
\underline{北原 鉄朗}: 
    ``旋律概形と筆圧感知を用いた作曲支援システム'', 
    {\it 情報処理学会 インタラクション2019(インタラクティブ発表),
    } 2A-05, March 2019. 

\item 
矢ヶ崎 里咲, 
\underline{北原 鉄朗}: 
    ``音楽がきっかけとなるコミュニケーション支援システム'', 
    {\it 情報処理学会 インタラクション2019(インタラクティブ発表),
    } 2B-40, March 2019. 

\item 
本間 直人, 
植村 あい子, 
\underline{北原 鉄朗}: 
    ``アニメの主題歌による類似アニメ検索の検討'', 
    {\it 情報処理学会 インタラクション2019(インタラクティブ発表),
    } 3B-53, March 2019. 

\item 
井上 湧哉, 
植村 あい子, 
\underline{北原 鉄朗}: 
    ``音楽と印象に関する一分析'', 
    {\it 情報処理学会研究報告(音楽情報科学),
    } Vol.2020-MUS-126, No.1, pp.1--17, February 2020. 

\item 
伊藤 健友, 
\underline{北原 鉄朗}: 
    ``カラオケにおける自動楽曲推薦'', 
    {\it インタラクション2020(インタラクティブ発表),
    } 1A-08, March 2020. 

\item 
山本 鷹人, 
\underline{北原 鉄朗}: 
    ``BGMの再生速度変化を用いた体幹トレーニング支援システム'', 
    {\it インタラクション2020(インタラクティブ発表),
    } 1A-09, March 2020. 

\item 
安坂 文汰, 
\underline{北原 鉄朗}: 
    ``動画の盛り上がり度に基づいたループシーケンサ'', 
    {\it インタラクション2020(インタラクティブ発表),
    } 2A-07, March 2020. 

\item 
草地 澪, 
植村 あい子, 
\underline{北原 鉄朗}: 
    ``ピアノ用自動バラード調アレンジシステム'', 
    {\it インタラクション2020(インタラクティブ発表),
    } 3A-09, March 2020. 

\item 
田原花蓮, 
植村あい子, 
北原鉄朗: 
    ``遺伝的アルゴリズムを用いたファミコン風自動編曲システムの生成'', 
    {\it 情報処理学会研究報告 音楽情報科学 (MUS),
    } Vol.2021-MUS-130(9), 2021. 

\item 
稲野友哉, 
北原鉄朗: 
    ``世代間ギャップの解消を目的とした楽曲再生システムの試作'', 
    {\it 情報処理学会研究報告 音楽情報科学 (MUS),
    } Vol.2021-MUS-130(41), 2021. 

\item 
新沼菫, 
饗庭絵里子, 
北原鉄朗: 
    ``BGM に含まれる言語が計算課題と読解課題に及ぼす影響'', 
    {\it 情報処理学会研究報告 音楽情報科学 (MUS),
    } Vol.2021-MUS-130(23), 2021. 

\item 
関晋之介, 
北原鉄朗: 
    ``ユーザの演奏のベロシティ変化を考慮するドラム演奏表情付けシステム'', 
    {\it 情報処理学会研究報告 音楽情報科学 (MUS),
    } Vol.2021-MUS-130(36), 2021. 

\item 
古庄優樹, 
北原鉄朗: 
    ``ギターの弦を正しく押さえるための初心者支援システム'', 
    {\it 情報処理学会研究報告 音楽情報科学 (MUS),
    } Vol.2021-MUS-130(35), 2021. 

\item 
廣岡彩笑, 
北原鉄朗: 
    ``スマートフォンを用いた合奏システムの試作'', 
    {\it 情報処理学会研究報告 音楽情報科学 (MUS),
    } Vol.2021-MUS-130(24), 2021. 

\item 
次田 直樹, 
\underline{北原 鉄朗}: 
    ``PC用キーボードを用いた演奏システムの試作'', 
    {\it 情報処理学会 第197回 ヒューマンコンピュータインタラクション研究会,
    } March 2022. 

\item 
杉浦 磨矢, 
\underline{北原 鉄朗}: 
    ``振りのタイミングを評価するダンス練習システム'', 
    {\it 情報処理学会 第197回 ヒューマンコンピュータインタラクション研究会,
    } March 2022. 

\item 
岩本 祐輝, 
\underline{北原 鉄朗}: 
    ``盛り上がり度に基づくループシーケンサにおけるユーザ適応の試み'', 
    {\it 情報処理学会 第197回 ヒューマンコンピュータインタラクション研究会,
    } March 2022. 

\end{Enumerate}

\section*{国内全国大会}
\begin{Enumerate}
  
\item 
\underline{北原 鉄朗}, 
後藤 真孝, 
奥乃 博: 
    ``楽器音オントロジー作成のための楽器音特徴抽出'', 
    {\it 情報処理学会 第62回全国大会,} 4M-5, March 2001. 

\item 
柳川 貴央, 
\underline{北原 鉄朗}, 
武田 正之: 
    ``即興演奏における演奏補正システム'', 
    {\it 情報処理学会 第64回全国大会,} 1L-5, March 2002. 

\item 
\underline{北原 鉄朗}, 
後藤 真孝, 
奥乃 博: 
    ``音色空間の音高依存性を考慮した楽器音の音源同定'', 
    {\it 日本音響学会2002年秋季研究発表会 講演論文集,} 1-1-4, pp.643--644, September 2002. 

\item 
\underline{北原 鉄朗}, 
後藤 真孝, 
奥乃 博: 
    ``音響的類似性に基づく楽器音の階層的クラスタリング'', 
    {\it 情報処理学会 第64回全国大会,} 1P-1, March 2003. 
{\bf (学生奨励賞)}
\item 
吉井 和佳, 
\underline{北原 鉄朗}, 
櫻庭
      洋平, 
奥乃 博: 
    ``教師なしクラスタリングと認識誤りパターンを利用した打楽器音の音源同定'', 
    {\it 情報処理学会 第64回全国大会,} 1P-3, March 2003. 

\item 
石田 克久, 
\underline{北原 鉄朗}, 
柳川 貴央, 
奥乃 博: 
    ``統計的アプローチに基づく即興演奏補正'', 
    {\it 情報処理学会 第64回全国大会,} 1P-3, March 2003. 

\item 
\underline{北原 鉄朗}, 
後藤 真孝, 
奥乃 博: 
    ``未知の楽器を考慮する楽器音の音源同定'', 
    {\it 情報処理学会 第66回全国大会,} 3ZA-3, March 2004. 
{\bf (学生奨励賞)}
\item 
吉岡 拓也, 
吉井 和佳, 
\underline{北原 鉄朗}, 
櫻庭
      洋平, 
尾形 哲也, 
奥乃 博: 
    ``音楽音響信号を対象とした和音変化時刻と和音名の同時認識'', 
    {\it 情報処理学会 第66回全国大会,} 3ZA-4, March 2004. 

\item 
石田 克久, 
\underline{北原 鉄朗}, 
武田 正之: 
    ``統計モデルに基づく旋律妥当性判定手法を用いた即興演奏支援'', 
    {\it 日本音響学会2004年秋季研究発表会 講演論文集,} 2-6-8, pp.783--784, September 2004. 

\item 
\underline{北原 鉄朗}, 
後藤 真孝, 
駒谷
      和範, 
尾形 哲也, 
奥乃 博: 
    ``多重奏の音源同定のための混合音からのテンプレート作成法'', 
    {\it 情報処理学会 第67回全国大会,} 3G-4, March 2005. 
{\bf (大会奨励賞)}
\item 
藤原
      弘将, 
\underline{北原 鉄朗}, 
後藤
      真孝, 
尾形 哲也, 
奥乃 博: 
    ``歌声の調波構造抽出を用いた歌手名の同定'', 
    {\it 情報処理学会 第67回全国大会,} 3R-8, March 2005. 

\item 
\underline{北原 鉄朗}, 
後藤 真孝, 
駒谷
      和範, 
尾形 哲也, 
奥乃 博: 
    ``混合音からの特徴量テンプレート作成と音楽的文脈の利用による多重奏の音源同定'', 
    {\it 日本音響学会2005年秋季研究発表会 講演論文集,} 3-10-15, September 2005. 

\item 
海尻 聡, 
石原 一志, 
\underline{北原 鉄朗}, 
Valin Jean-Marc, 
駒谷
      和範, 
尾形 哲也, 
奥乃 博: 
    ``ロボットによる周囲状況把握のための雑音下での環境音認識'', 
    {\it 計測自動制御学会 第6回システムインテグレーション部門講演会 (SI2005),
    } December 2005. 

\item 
糸山 克寿, 
\underline{北原 鉄朗}, 
駒谷
      和範, 
尾形 哲也, 
奥乃 博: 
    ``多重奏中特定パートの自動採譜における複数特徴量の自動重み付け'', 
    {\it 情報処理学会 第68回全国大会,} 2L-6, March 2006. 

\item 
西山 正紘, 
\underline{北原 鉄朗}, 
駒谷
      和範, 
尾形 哲也, 
奥乃 博: 
    ``標題音楽アノテーションのための階層的物語タグの設計'', 
    {\it 情報処理学会 第68回全国大会,} 3L-6, March 2006. 

\item 
田口 明裕, 
\underline{北原 鉄朗}, 
石原 一志, 
駒谷
      和範, 
尾形 哲也, 
奥乃 博: 
    ``擬音語表現を利用した環境音のためのXMLタグの設計と自動付与'', 
    {\it 情報処理学会 第68回全国大会,} 3L-7, March 2006. 

\item 
\underline{北原 鉄朗}, 
後藤
      真孝, 
駒谷
      和範, 
尾形 哲也, 
奥乃 博: 
    ``Instrogram:楽器存在確率に基づく音楽視覚表現法'', 
    {\it 日本音響学会2006年春季研究発表会 講演論文集,} 2-2-13, March 2006. 

\item 
藤原
      弘将, 
\underline{北原 鉄朗}, 
後藤
      真孝, 
駒谷
      和範, 
尾形 哲也, 
奥乃 博: 
    ``調波構造抽出と高信頼度フレーム選択を用いた雑音下での話者識別'', 
    {\it 日本音響学会2006年春季研究発表会 講演論文集,} 1-11-17, March 2006. 

\item 
\underline{北原 鉄朗}, 
後藤
      真孝, 
駒谷
      和範, 
尾形 哲也, 
奥乃 博: 
    ``Instrogramを用いた類似楽曲検索'', 
    {\it 日本音響学会2006年秋季研究発表会 講演論文集,} 2-7-1, September 2006. 

\item 
西山 正紘, 
\underline{北原 鉄朗}, 
駒谷
      和範, 
尾形 哲也, 
奥乃 博: 
    ``マルチメディアコンテンツにおける音楽と映像の調和に関する分析'', 
    {\it 情報処理学会 第70回全国大会,} 2N-6, March 2007. 

\item 
清水 敬太, 
\underline{北原 鉄朗}, 
駒谷
      和範, 
尾形 哲也, 
奥乃 博: 
    ``OnomaTree:擬音語と木構造を併用した環境音検索インターフェース'', 
    {\it 情報処理学会 第69回全国大会,} 3N-7, March 2007. 

\item 
\underline{北原 鉄朗}, 
橋田 光代, 
片寄 晴弘: 
    ``音楽情報処理のための共通データフォーマットCrestMuseXML−全体構想と基本設計方針−'', 
    {\it 日本音響学会2007年秋季研究発表会 講演論文集,} 2-1-4, September 2007. 

\item 
\underline{北原 鉄朗}, 
後藤
      真孝, 
奥乃 博, 
片寄 晴弘: 
    ``Instrogram:多重奏中の楽器構成に関する確率論的表現法'', 
    {\it 電子情報通信学会2008年総合大会,
    } AS-5-4, March 2008. 

\item 
風谷 真志, 
\underline{北原 鉄朗}, 
片寄 晴弘: 
    ``確率文脈自由文法を用いた事例参照型自動作曲システム'', 
    {\it 情報処理学会 第70回全国大会,} 3X-3, March 2008. 

\item 
小林 一樹, 
\underline{北原 鉄朗}: 
    ``効率的なロボットプログラミング環境の実現に向けて'', 
    {\it 第22回人工知能学会全国大会,} 2G1-1, May 2008. 

\item 
\underline{北原 鉄朗}, 
片寄 晴弘: 
    ``MIDIデータのベロシティを異なる音源に適応させる試み'', 
    {\it 日本音響学会2008年秋季研究発表会 講演論文集,} 1-9-16, September 2008. 

\item 
山川 暢英, 
\underline{北原 鉄朗}, 
高橋 徹, 
駒谷 和範, 
尾形 哲也, 
奥乃 博: 
    ``環境音から擬音語への自動変換における特徴量抽出法の検討'', 
    {\it 情報処理学会第72回全国大会,
    } 3U-9, March 2010. 

\item 
水本 直希, 
\underline{北原 鉄朗}, 
片寄 晴弘: 
    ``エレキギターにおける演奏情報の特徴抽出'', 
    {\it 情報処理学会第72回全国大会,
    } 5T-1, March 2010. 

\item 
村主 大輔, 
森勢 将雅, 
\underline{北原 鉄朗}, 
片寄 晴弘: 
    ``奄美大島民謡風歌声合成のためのコブシに着目した歌声の特徴分析'', 
    {\it 情報処理学会第72回全国大会,
    } 6U-4, March 2010. 

\item 
山川 暢英, 
高橋 徹, 
\underline{北原 鉄朗}, 
尾形 哲也, 
奥乃 博: 
    ``ロボット聴覚のための Matching-Pursuit による環境音の分離音認識'', 
    {\it 日本ロボット学会第28回学術講演会,
    } 1H2-4, September 2010. 

\item 
山川 暢英, 
\underline{北原 鉄朗}, 
高橋 徹, 
尾形 哲也, 
奧乃 博: 
    ``ロボット聴覚のためのMatching Pursuitによる複数環境音の同定'', 
    {\it 情報処理学会 第73回全国大会講演論文集,
    } 6P-3, March 2011. 

\item 
山川 暢英, 
\underline{北原 鉄朗}, 
高橋 徹, 
尾形 哲也, 
奥乃 博: 
    ``擬音語と環境音の音響的関係性を考慮した環境音to擬音語変換システム'', 
    {\it 2011年度人工知能学会全国大会,
    } 1C2-OS4b-4, June 2011. 

\item 
松本 大希, 
滝口 恭平, 
小高 大典, 
\underline{北原 鉄朗}: 
    ``複数人が共有する場のためのBGM選曲手法の検討'', 
    {\it 日本音響学会2012年春季研究発表会講演論文集,
    } 2-6-8, March 2012. 

\item 
土屋 裕一, 
\underline{北原 鉄朗}: 
    ``旋律編集の一手法'', 
    {\it 日本音響学会2012年春季研究発表会講演論文集,
    } 3-6-14, March 2012. 

\item 
鈴木 峻平, 
竹内 俊雄, 
佐藤 桂亮, 
\underline{北原 鉄朗}: 
    ``確率推論を用いた四声体和声の自動生成'', 
    {\it 日本音響学会2012年春季研究発表会論文集,
    } 3-6-15, March 2012. 

\item 
木村 翔平, 
鈴木 優, 
鈴木 智文, 
\underline{北原 鉄朗}: 
    ``音楽理論に基づいた鼻歌作曲支援システム“ハミコン”'', 
    {\it 日本音響学会2012年春季研究発表会講演論文集,
    } 3-6-16, March 2012. 

\item 
松方 翔吾, 
\underline{北原 鉄朗}: 
    ``トランペット演奏時における口唇周囲の筋活動と音響情報の関係性について'', 
    {\it 日本音響学会2013年春季研究発表会講演論文集,
    } 1-1-7, March 2013. 

\item 
大塚 匡紀, 
\underline{北原 鉄朗}: 
    ``ギター演奏者のためのリアルタイムベースライン生成システム'', 
    {\it 日本音響学会2013年春季研究発表会講演論文集,
    } 1-1-14, March 2013. 

\item 
山下 雄史, 
岡田 美咲, 
\underline{北原 鉄朗}: 
    ``手書き入力によって盛り上がりをコントロールするループシーケンサ'', 
    {\it 日本音響学会2013年春季研究発表会講演論文集,
    } 1-1-16, March 2013. 

\item 
鈴木 峻平, 
\underline{北原 鉄朗}: 
    ``確率推論を用いた四声体和声の自動生成 \&\#65293;讃美歌データベースによる実験結果の報告\&\#65293;'', 
    {\it 日本音響学会2013年春季研究発表会講演論文集,
    } 1-1-8, March 2013. 

\item 
帆苅 隼佑, 
長安 達也, 
\underline{北原 鉄朗}: 
    ``ジョギングのペースに再生速度を同期させるスマートフォン用音楽プレイヤー'', 
    {\it 情報処理学会第75回全国大会,
    } 4W-4, March 2013. 

\item 
岡田 風由子, 
後藤 駿典, 
小林 一樹, 
\underline{北原 鉄朗}: 
    ``ロボットを用いたメッセージ着信通知の一手法'', 
    {\it 情報処理学会第75回全国大会,
    } 6ZA-2, March 2013. 
{\bf (学生奨励賞受賞)}
\item 
山内 雅史, 
篠本 亮, 
\underline{北原 鉄朗}: 
    ``Kinectを用いたダンス学習支援システムの開発'', 
    {\it 情報処理学会第75回全国大会,
    } 2ZG-7, March 2013. 
{\bf (学生奨励賞受賞)}
\item 
\underline{北原 鉄朗}: 
    ``CrestMuse Toolkit: ロボット聴覚ソフトウェアHARKとの連携'', 
    {\it 電子情報通信学会2013年総合大会,
    } D-14-12, March 2013. 

\item 
吉永 眞宏, 
鈴木 光, 
\underline{北原 鉄朗}: 
    ``Kinectを用いた音源定位の性能評価'', 
    {\it 電子情報通信学会2013年総合大会(ISS学生ポスターセッション),
    } ISS-P-268, March 2013. 

\item 
松方 翔吾, 
寺澤 洋子, 
松原 正樹, 
\underline{北原 鉄朗}: 
    ``トランペット演奏時における音響特徴量から筋活動量への変換'', 
    {\it 日本音響学会2014年春季研究発表会講演論文集,
    } 1-5-2, March 2014. 

\item 
土屋 裕一, 
\underline{北原 鉄朗}: 
    ``旋律概形を用いた作曲支援システム:ユーザビリティ実験の報告'', 
    {\it 情報処理学会第76回全国大会,
    } 1R-2, March 2014. 
{\bf (学生奨励賞受賞)}
\item 
鈴木 峻平, 
\underline{北原 鉄朗}: 
    ``ベイジアンネットワークを用いた四声体和声付け:コードノードの有無による出力結果の比較'', 
    {\it 情報処理学会第76回全国大会,
    } 2R-2, March 2014. 
{\bf (学生奨励賞受賞)}
\item 
小林 彩夏, 
林  義久, 
中根 晴香, 
\underline{北原 鉄朗}: 
    ``マッシュアップ音楽作成のための選曲支援の一検討'', 
    {\it 情報処理学会第76回全国大会,
    } 6R-3, March 2014. 
{\bf (学生奨励賞受賞)}
\item 
鈴木  光, 
吉永 眞宏, 
小暮 計貴, 
\underline{北原 鉄朗}: 
    ``雑音環境下のための音声案内システム:周囲の雑音レベルに合わせた音量の自動調整'', 
    {\it 情報処理学会第76回全国大会,
    } 6S-1, March 2014. 

\item 
小暮 計貴, 
吉永 眞宏, 
鈴木  光, 
\underline{北原 鉄朗}: 
    ``雑音環境下のための音声案内システム:ユーザの聞き返しに基づく音量の自動調整'', 
    {\it 情報処理学会第76回全国大会,
    } 6S-2, March 2014. 

\item 
西脇 絵里子, 
小野澤 理紗, 
\underline{北原 鉄朗}: 
    ``ユーザーの習熟度に合わせた初心者向けダンス学習支援システム'', 
    {\it 情報処理学会第76回全国大会,
    } 4ZD-7, March 2014. 

\item 
\underline{北原 鉄朗}, 
土屋 裕一: 
    ``旋律を簡約・操作する一手法'', 
    {\it 人工知能学会第28回全国大会,
    } 1K4-OS-07a-3, May 2014. 

\item 
小暮計貴, 
\underline{北原 鉄朗}: 
    ``周囲の雑音に基づき音量調節を行う音声対話システム:セミブラインド音源分離の導入の検討'', 
    {\it 情報処理学会第77回全国大会,
    } 5P-04, March 2015. 

\item 
飯島孔右, 
鶴岡亜也佳, 
\underline{北原 鉄朗}: 
    ``手書き入力で盛り上がりをコントロールするループシーケンサ:スペクトログラムからの盛り上がり度の自動割り振り'', 
    {\it 情報処理学会第77回全国大会,
    } 2S-03, March 2015. 

\item 
鈴木潤一, 
末次尚之, 
\underline{北原 鉄朗}: 
    ``複数ユーザー間での楽曲推薦を実現するミュージックプレイヤー'', 
    {\it 情報処理学会第77回全国大会,
    } 3S-04, March 2015. 
{\bf (学生奨励賞受賞)}
\item 
木下尚洋, 
栗原拓也, 
山口竜之介, 
\underline{北原 鉄朗}: 
    ``カラオケを盛り上げるためのタンバリン演奏支援システム'', 
    {\it 情報処理学会第77回全国大会,
    } 4P-03, March 2015. 

\item 
大塚匡紀, 
\underline{北原 鉄朗}: 
    ``MIDIギターと音響信号処理の統合によるギター演奏の自動採譜の検討'', 
    {\it 情報処理学会第77回全国大会,
    } 5S-02, March 2015. 

\item 
\underline{北原 鉄朗}: 
    ``音符表現によらない旋律の木構造表現の予備検討'', 
    {\it 人工知能学会第29回全国大会,
    } 2C4-OS-21a-2, May 2015. 

\item 
小泉 遼, 
岩崎 順, 
長村佳祐, 
\underline{北原 鉄朗}: 
    ``カラオケにおける音痴の改善支援のための予備調査'', 
    {\it 情報処理学会 第78回全国大会,
    } 1Q-06, March 2016. 
{\bf (学生奨励賞受賞)}
\item 
栗原拓也, 
横溝有希子, 
竹腰美夏, 
馬場哲晃, 
\underline{北原 鉄朗}: 
    ``カラオケを盛り上げるためのタンバリン演奏支援システム:複数人プレイへの対応'', 
    {\it 情報処理学会 第78回全国大会,
    } 1Q-07, March 2016. 

\item 
大内彬裕, 
\underline{北原 鉄朗}: 
    ``ギター弾き語り演奏を入力した自動編曲システムの試作'', 
    {\it 情報処理学会 第78回全国大会,
    } 2Q-07, March 2016. 

\item 
中條早織, 
豊田裕也, 
\underline{北原 鉄朗}: 
    ``Androidを用いた演劇支援のためのUnity3Dアプリケーションの開発'', 
    {\it 情報処理学会 第78回全国大会,
    } 6Z-05, March 2016. 

\item 
高屋敷弓恵, 
棚橋 徹, 
\underline{北原 鉄朗}: 
    ``面接技能向上のための自己PR支援システム'', 
    {\it 情報処理学会 第78回全国大会,
    } 6X-02, March 2016. 

\item 
鈴木潤一, 
\underline{北原 鉄朗}: 
    ``友人同士で好みの楽曲を聴かせ合うスマートフォン用ミュージックプレイヤー:楽曲推薦手法の一改善'', 
    {\it 情報処理学会 第78回全国大会,
    } 6B-03, March 2016. 

\item 
大野 涼平, 
森勢 将雅, 
\underline{北原 鉄朗}: 
    ``アニメ風音声への加工のための韻律分析'', 
    {\it 日本音響学会 2016年春季研究発表会 講演論文集,
    } 3-P-30, March 2016. 

\item 
佐藤 愛, 
\underline{北原 鉄朗}, 
寺澤 洋子, 
松原 正樹: 
    ``トランペット演奏時の口唇周囲および腹部の筋活動と音響的特徴の関係'', 
    {\it 日本音響学会 2016年春季研究発表会 講演論文集,
    } 2-10-7, March 2016. 

\item 
\underline{北原 鉄朗}: 
    ``音符表現によらない旋律の木構造表現の検討(第2報)'', 
    {\it 人工知能学会第30回全国大会(JSAI2016),
    } 3G3-OS-15a-3, June 2016. 

\item 
平田 圭二, 
大村 英史, 
\underline{北原 鉄朗}: 
    ``旋律の微分と簡約の導入'', 
    {\it 人工知能学会第30回全国大会(JSAI2016),
    } 3G3-OS-15a-4, June 2016. 

\item 
長谷川 翔太, 
大野 涼平, 
\underline{北原 鉄朗}: 
    ``男性両声類の女声らしさに関わる特徴量の分析'', 
    {\it 日本音響学会 2017年春季研究発表会 講演論文集,
    } 1-Q-39, March 2017. 

\item 
大野 涼平, 
高道 慎之介, 
森勢 将雅, 
\underline{北原 鉄朗}: 
    ``統計的声質変換における印象変化の調査'', 
    {\it 日本音響学会 2017年春季研究発表会 講演論文集,
    } 2-P-39, March 2017. 

\item 
松浦 佳輝, 
棚橋 徹, 
\underline{北原 鉄朗}: 
    ``パターン認識を用いた特定のベーシストの特徴の分析'', 
    {\it 情報処理学会 第79回全国大会,
    } 3L-01, March 2017. 

\item 
島田 彩女, 
松村 ひかる, 
森尻 有貴, 
\underline{北原 鉄朗}: 
    ``ピアノ練習支援のための楽譜表示システムの試作'', 
    {\it 情報処理学会 第79回全国大会,
    } 5L-05, March 2017. 
{\bf (学生奨励賞 受賞)}
\item 
棚橋 徹, 
小林 一樹, 
\underline{北原 鉄朗}: 
    ``音によるヒューマン・エージェント・インタラクションのためのプロトタイプシステムの試作'', 
    {\it 情報処理学会 第79回全国大会,
    } 6Y-01, March 2017. 
{\bf (学生奨励賞 受賞)}
\item 
水野 創太, 
白松 俊, 
一ノ瀬 修吾, 
\underline{北原 鉄朗}: 
    ``即興合奏支援システムのためのスマートフォンセンサーを用いた身体動作認識手法'', 
    {\it 情報処理学会 第79回全国大会,
    } 2ZA-04, March 2017. 

\item 
山田 竜郎, 
\underline{北原 鉄朗}, 
有江 浩明, 
尾形 哲也: 
    ``LSTMを用いた四声体和声の生成'', 
    {\it 人工知能学会第31回全国大会(JSAI 2017),
    } 2C3-OS-20a-1, May 2017. 

\item 
水野 創太, 
一ノ瀬 修吾, 
白松 俊, 
\underline{北原 鉄朗}: 
    ``演奏未経験者のためのスマートフォンセンサーを用いた即興合奏支援システムの試作'', 
    {\it 人工知能学会第31回全国大会(JSAI 2017),
    } 2C3-OS-20a-4, May 2017. 

\item 
平田 圭二, 
伊藤 貴之, 
\underline{北原 鉄朗}, 
深山 覚, 
今井 慎太郎, 
持橋 大地: 
    ``パネルディスカッション:人工知能は作曲家/演奏家になれるか?'', 
    {\it 人工知能学会第31回全国大会(JSAI 2017),
    } 2C4-OS-20b-3, May 2017. 

\item 
大野 涼平, 
高道 慎之介, 
森勢 将雅, 
\underline{北原 鉄朗}: 
    ``音声の「かわいさ」における主観的傾向の一分析'', 
    {\it 日本音響学会2017年秋季研究発表会講演論文集,
    } 3-P-32, September 2017. 

\item 
宇田川真唯, 
植村 あい子, 
\underline{北原 鉄朗}: 
    ``ギター初心者のための演奏練習支援システムの提案'', 
    {\it 情報処理学会 第80回全国大会 講演論文集,
    } 1N-1, March 2018. 

\item 
白石 美南, 
小笠原 梢, 
\underline{北原 鉄朗}: 
    ``カラオケのためのハモリパート練習システム〜ハモリパートの自動生 成および練習支援システムの試作〜'', 
    {\it 情報処理学会 第80回全国大会 講演論文集,
    } 1N-2, March 2018. 

\item 
小笠原 梢, 
白石 美南, 
\underline{北原 鉄朗}: 
    ``カラオケのためのハモリパート練習支援システム〜試作システムを用
いた被験者実験の報告〜'', 
    {\it 情報処理学会 第80回全国大会 講演論文集,
    } 1N-3, March 2018. 
{\bf (学生奨励賞受賞)}
\item 
蓮井 星良, 
石山 俊之, 
\underline{北原 鉄朗}: 
    ``HMD を用いたヴァーチャルなドラム演奏環境の試作'', 
    {\it 情報処理学会 第80回全国大会 講演論文集,
    } 1N-5, March 2018. 

\item 
甚野 健太, 
大野 涼平, 
\underline{北原 鉄朗}: 
    ``合いの手「PPPH」が入る楽曲の特徴に関する一分析'', 
    {\it 情報処理学会 第80回全国大会 講演論文集,
    } 1N-9, March 2018. 

\item 
草野 有沙, 
西由 佳梨, 
\underline{北原 鉄朗}: 
    ``読書を促進する音楽付き読書アプリの提案'', 
    {\it 情報処理学会 第80回全国大会 講演論文集,
    } 7Y-1, March 2018. 
{\bf (学生奨励賞受賞)}
\item 
棚橋 徹, 
小林 一樹, 
\underline{北原 鉄朗}: 
    ``言語情報を持たない音を用いたヒューマン・エージェント・インタラ クションシステムの開発 '', 
    {\it 情報処理学会 第80回全国大会 講演論文集,
    } 6ZA-6, March 2018. 

\item 
松下 禎希, 
大野 涼平, 
\underline{北原 鉄朗}: 
    ``雑音が記憶や作業に与える影響に関する一調査'', 
    {\it 日本音響学会 2018年春季研究発表会 講演論文集,
    } 2-P-11, March 2018. 

\item 
伊藤 春菜, 
棚橋 徹, 
\underline{北原 鉄朗}: 
    ``イビキ音のマスキングに関する一検討'', 
    {\it 日本音響学会 2018年春季研究発表会 講演論文集,
    } 2-P-16, March 2018. 

\item 
大野 涼平, 
高道 慎之介, 
森勢 将雅, 
\underline{北原 鉄朗}: 
    ``話者適応型 RBM を用いたユーザが所望するかわいい音声への声質変換'', 
    {\it 日本音響学会 2018年春季研究発表会 講演論文集,
    } 2-Q-33, March 2018. 

\item 
志賀 あゆみ, 
\underline{北原 鉄朗}: 
    ``ジャズのベースラインの自動生成'', 
    {\it 日本音響学会 2019年春季研究発表会 講演論文集,
    } 3-1-1, March 2019. 

\item 
植村 あい子, 
\underline{北原 鉄朗}: 
    ``コードモーフィングに基づくリハーモナイゼーションの一検討'', 
    {\it 日本音響学会 2019年春季研究発表会 講演論文集,
    } 3-1-5, March 2019. 

\item 
増田 誠也, 
饗庭 絵里子, 
\underline{北原 鉄朗}: 
    ``グラフィックイコライザによる音色操作と印象との関係'', 
    {\it 日本音響学会 2019年春季研究発表会 講演論文集,
    } 3-1-7, March 2019. 

\item 
河村 翔太, 
植村 あい子, 
\underline{北原 鉄朗}: 
    ``歌詞と音楽が与える印象の分析'', 
    {\it 日本音響学会 2019年春季研究発表会 講演論文集,
    } 3-1-9, March 2019. 

\item 
島村 美羽, 
阿久井 愛, 
\underline{北原 鉄朗}: 
    ``多人数演奏楽譜から連弾譜への自動編曲'', 
    {\it 日本音響学会 2022年 春季研究発表会,
    } March 2022. 

\item 
阿久井 愛, 
島村 美羽, 
\underline{北原 鉄朗}: 
    ``Jpopのラテン風ピアノ編曲システムの試作'', 
    {\it 日本音響学会 2022年 春季研究発表会,
    } March 2022. 

\item 
山川 峻, 
\underline{北原 鉄朗}: 
    ``GTTM分析のオープンソース実装'', 
    {\it 日本音響学会 2022年 春季研究発表会,
    } March 2022. 

\item 
大下 沙偉, 
\underline{北原 鉄朗}: 
    ``フルート練習システム構築のための音響分析'', 
    {\it 日本音響学会 2022年 春季研究発表会,
    } March 2022. 

\end{Enumerate}

\section*{口頭発表}
\begin{Enumerate}
  
\item 
\underline{北原 鉄朗}, 
後藤
      真孝, 
奥乃 博: 
    ``音高による音色変化と未知楽器の問題を考慮した楽器音の音源同定'', 
    {\it 日本音響学会関西支部 第6回若手研究者交流研究発表会,
    } December 2003. 

\item 
\underline{北原 鉄朗}, 
後藤
      真孝, 
駒谷
      和範, 
尾形 哲也, 
奥乃 博: 
    ``多重奏の音源同定における音の重なりに対する頑健性の改善'', 
    {\it 日本音響学会関西支部 第8回若手研究者交流研究発表会,
    } December 2005. 
{\bf (若手奨励賞受賞)}
\item 
関 晋之介, 
井上 湧哉, 
\underline{北原 鉄朗}: 
    ``ドラム演奏表情付けに向けたドラム演奏のベロシティの分析'', 
    {\it 情報処理学会 第126回音楽情報科学研究会 萌芽・デモ・議論セッション,
    } February 2020. 

\item 
塚本 康太, 
饗庭 絵里子, 
\underline{北原 鉄朗}: 
    ``クラシック曲の楽譜データに対する自動和声付与システムの構築に向けて'', 
    {\it 情報処理学会 第126回音楽情報科学研究会 萌芽・デモ・議論セッション,
    } February 2020. 

\end{Enumerate}

\section*{解説記事}
\begin{Enumerate}
  
\item 
奥乃 博, 
\underline{北原 鉄朗}, 
吉井 和佳: 
    ``楽曲の特徴量抽出と検索技術'', 
    {\it 電気学会誌,
    } 特集「音響機器は進歩している」, Vol.127, No.7, pp.417--420, July 2007. 

\item 
\underline{北原 鉄朗}: 
    ``音楽情報処理最前線! 楽器で音楽が探せる 「楽器認識技術」が叶える音楽の新しい聴き方・探し方'', 
    {\it DTM Magazine,
    } Vol.176, pp.102--103, February 2009. 

\item 
平井 重行, 
橋田 光代, 
\underline{北原 鉄朗}, 
竹川 佳成, 
片寄 晴弘: 
    ``音楽とヒューマンインタフェース'', 
    {\it 情報処理,
    } 特集「音楽処理技術の最前線」, Vol.50, No.8, pp.756--763, August 2009. 

\item 
\underline{北原 鉄朗}: 
    ``私のブックマーク「音楽情報処理」'', 
    {\it 人工知能学会誌,
    } Vol.24, No.5, pp.921--929, November 2009. 

\item 
\underline{北原 鉄朗}: 
    ``BOOK REVIEW: 音楽はなぜ心に響くのか---音楽音響学と音楽を解き明かす諸科学---'', 
    {\it 日本バーチャル・リアリティ学会誌,
    } Vol.17, No.4, pp.268, December 2012. 

\item 
\underline{北原 鉄朗}, 
深山 覚: 
    ``自動作曲・自動編曲の現状と課題'', 
    {\it 電子情報通信学会誌,
    } 小特集「音楽情報処理技術:分析から合成・作曲・利活用まで」, Vol.98, No.6, pp.475--479, June 2015. 

\item 
\underline{北原 鉄朗}, 
永野 秀尚: 
    ``特集「音楽を軸に拡がる情報科学」編集にあたって'', 
    {\it 情報処理(情報処理学会誌),
    } Vol.57, No.6, pp.504--505, June 2016. 

\end{Enumerate}

    \section*{章分担}
    \begin{Enumerate}
    
\item 
\underline{Tetsuro Kitahara}: 
    ``Mid-level Representations of Musical Audio Signals for Music Information Retrieval'', 
    {\it Advances in Music Information Retrieval,
    } Studies in Computational Intelligence 274, (Zbigniew W. Ras and Alicja A. Wieczorkowska (Eds.)), Springer, February 2010. 

\item 
\underline{北原 鉄朗}: 
    ``自動採譜'', 
    {\it 音響キーワードブック,
    } (日本音響学会 (Eds.)), コロナ社, March 2016. 

\item 
\underline{北原 鉄朗}: 
    ``楽器音の特徴と識別'', 
    {\it 音楽知覚認知ハンドブック,
    } 7.4.6節, 北大路書房, February 2020. 

\item 
\underline{北原 鉄朗}: 
    ``レンダリングシステム'', 
    {\it 音楽知覚認知ハンドブック,
    } 7.6.3節, 北大路書房, February 2020. 

    \end{Enumerate}
  
\section*{翻訳}
\begin{Enumerate}
  
\item 
Francois Pachet(著), 
北原 鉄朗(訳): 
    ``デジタル音楽配信のためのコンテンツ管理'', 
    {\it Communications of the ACM 日本語版,
    } Vol.4, No.2, pp.1--6, June 2004. 

\item 
Bryan Pardo(著), 
北原 鉄朗(訳): 
    ``音楽情報検索'', 
    {\it Communications of the ACM 日本語版,
    } Vol.6, No.2, pp.1--3, 2007. 

\item 
Avery Wang(著), 
北原 鉄朗(訳): 
    ``Shazam 音楽認識サービス'', 
    {\it Communications of the ACM 日本語版,
    } Vol.6, No.2, pp.17--21, 2007. 

\end{Enumerate}

   \section*{招待講演・パネルディスカッションなど}
   \begin{itemize}
   
\item 

    ``パネルディスカッション「音楽情報処理研究者\{に,が\}望むこと」'', 
    {\it 情報処理学会第51回音楽情報科学研究会,
    } パネリスト, August 2003. 

\item 

    ``新博士によるパネルディスカッション1「博士への道のりと将来への夢」'', 
    {\it 情報処理学会第71回音楽情報科学研究会,
    } パネリスト, August 2007. 

\item 

    ``パネルディスカッション「``音''研究の未来」'', 
    {\it 情報処理学会 音楽情報科学研究会・音声言語情報処理研究会 特別合同企画,
    } パネリスト, February 2008. 

\item 

    ``音楽の信号処理とパターン処理の基礎技術:入門と実践'', 
    {\it 情報処理学会 第76回音楽情報科学研究会 チュートリアル,
    } 講師, August 2008. 

\item 

    ``パネルディスカッション:作るだけでいいの?調べるだけでいいの?'', 
    {\it 情報処理学会第78回音楽情報科学研究会・日本音響学会音楽音響研究会 合同特別企画,
    } パネリスト, December 2008. 

\item 

    ``CrestMuseXML Toolkitで始める音楽情報処理入門'', 
    {\it 情報処理学会 第80回音楽情報科学研究会 チュートリアル,
    } 講師, May 2009. 

\item 
\underline{北原 鉄朗}: 
    ``計算機を用いた音楽創作支援の現状'', 
    {\it 計測自動制御学会 先端電子計測部会講演会,
    } 招待講演, November 2014. 

\item 
\underline{北原 鉄朗}: 
    ``大学で働く研究者'', 
    {\it 日本音響学会2015年春季研究発表会 ビギナーズセミナー,
    } March 2015. 

\item 

    ``パネルディスカッション「エンターテインメントを深化させる音楽情報処理研究」'', 
    {\it Computer Entertainment Developers Conference (CEDEC 2015),
    } オーガナイザー兼パネリスト, August 2015. 

\item 

    ``人工知能は作曲家/演奏家になれるか?'', 
    {\it 2017年人工知能学会全国大会,
    } パネリスト, May 2017. 

\item 
北原鉄朗: 
    ``誰もが創作を通じて音楽を楽しめる世界を目指して'', 
    {\it 音学シンポジウム2021,
    } June 2021. 
(招待講演)
\item 
北原鉄朗: 
    ``北原研究室の研究事例紹介:ベーシストの旋律分析とイコライザーの印象分析'', 
    {\it Music×Analytics Meetup \#5,
    } June 2021. 

   \end{itemize}
 
\section*{特許}
\begin{itemize}
  
\item 
鍵盤楽器支援装置及び鍵盤楽器支援システム,特開2006-145681号(2006年6月8日),特願2004-333279(2004年11月17日),発明者:武田 正之,石田
      克久,北原 鉄朗.\par

\item 
楽器音認識方法,楽器アノテーション方法,及び楽曲検索方法,特願2006-058649号(2006年3月3日),特開2007-240552号(2007年9月20日),発明者:北原
      鉄朗,奥乃 博. \par

\item 
旋律編集装置,旋律編集方法及び旋律編集プログラム,特願2012-046062号(2012年3月2日),発明者: 北原 鉄朗,土屋 裕一.\par

\end{itemize}

\section*{助成金}
\begin{itemize}
  
\item 
(財)C\&C振興財団 国際会議論文発表者助成 採択(IEA/AIE-2003での発表に対して)\par

\item 
(財)情報科学国際交流財団 研究者海外派遣助成 採択(ICME 2003での発表に対して)\par

\item 
(財)原総合知的通信システム基金 国際会議論文発表助成 採択(ICASSP 2004での発表に対して)\par

\item 
(財)電気通信普及財団 海外渡航旅費援助 採択(ISMIR 2005での発表に対して)\par

\item 
(財)立石科学技術振興財団 国際交流助成 採択(ICASSP 2006での発表に対して)\par

\item 
(財)電気通信普及財団 海外渡航旅費援助 採択(ISM 2008での発表に対して)\par

\item 
平成15年度 ASTEM学生ベンチャー奨励金制度 奨励金採択\par
「即興演奏の不自然な旋律を自動的に補正する機能を組み込んだ電子楽器の開発」\par

\item 
平成16年度 SCAT研究奨励金 採択\par
「音楽音響信号に対するMPEG-7タグの自動付与および音楽情報検索への応用」\par

\item 
21世紀COE「知識社会基盤構築のための情報学拠点形成」平成16年度 若手リーダーシップ養成プログラム研究費 採択\par
「高度な音楽検索実現のための音楽音響信号に対するMPEG-7タグの自動付与」\par

\item 
日本学術振興会 科学研究費補助金 特別研究員研究奨励費(平成17\&\#65374;18年度)\par
「音楽のディジタルアーカイブ化のためのMPEG-7タグの設計と自動付与」\par

\item 
日本学生支援機構 第1種奨学金「特に優れた業績による返還免除」認定(全額)\par

\item 
平成24年度 SCAT研究助成 採択 
「円滑なヒューマン・ロボット・コミュニケーションのための
相手の反応に基づく振る舞いのリアルタイム適応」,研究代表者:北原 鉄朗.\par

\item 
平成24年度 総務省 戦略的情報通信研究開発推進制度(SCOPE)若手ICT研究者等育成型研究開発 採択
「対話相手の状況をリアルタイムに推定して自身の挙動を適応させる音声対話ロボットの研究開発」,研究代表者:北原 鉄朗.\par

\item 
日本学術振興会 科学研究費助成事業 若手研究B(平成26〜27年度)「金管楽器演奏に対する音響空間と筋電空間の相互マッピング」(研究代表者)\par

\item 
日本学術振興会 科学研究費助成事業 基盤研究A(平成26〜28年度)「音楽の作曲・演奏・信号の数理モデルを融合する音楽音響情報処理の研究」(研究分担者)(代表:嵯峨山 茂樹)\par

\item 
日本学術振興会 科学研究費補助事業 基盤研究B(平成26〜28年度)「木構造に基づく時系列メディアの表現法の提案とその操作系の実現」(研究分担者)(代表:平田 圭二)\par

\item 
日本学術振興会 科学研究費補助事業 若手研究B(平成28〜30年度)「相互予測によるコミュニケーションの相互適応モデルの構築と音楽演奏を用いた検証」(研究代表者)\par

\item 
日本学術振興会 科学研究費補助事業 基盤研究A(平成28〜32年度)「統計的文法理論と構成的意味論に基づく音楽理解の計算モデル」(研究分担者)(代表:東条 敏)\par

\end{itemize}

\section*{受賞}
\begin{itemize}
  
  \item 
  情報処理学会 第64回全国大会学生奨励賞
  \item 
  電気通信普及財団 第19回テレコムシステム技術学生賞 受賞
  \item 
  情報処理学会 第66回全国大会学生奨励賞
  \item 
  情報処理学会 第67回全国大会大会奨励賞
  \item 
  IEEE関西支部 第3回学生研究奨励賞 受賞
  \item 
  第3回IPSJ Digital Courier船井若手奨励賞
  \item 
  第2回京都大学総長賞受賞
  \item 
  日本音響学会関西支部 第8回若手研究者交流研究発表会若手奨励賞受賞
  \item 
  学生奨励賞受賞
  \item 
  学生奨励賞受賞
  \item 
  学生奨励賞受賞
  \item 
  学生奨励賞受賞
  \item 
  学生奨励賞受賞
  \item 
  学生奨励賞受賞
  \item 
  学生奨励賞受賞
  \item 
  学生奨励賞受賞
  \item 
  学生奨励賞 受賞
  \item 
  学生奨励賞 受賞
  \item 
  Best Student Paper Award
  \item 
  学生奨励賞受賞
  \item 
  学生奨励賞受賞
\end{itemize}

\section*{学会活動}
\begin{itemize}
  
\item 
情報処理学会 音楽情報科学研究会 主査(2015年度〜2016年度)\par

\item 
2016年度人工知能学会全国大会(第30回) 大会委員\par

\item 
Special Session on Music Information Processing, the IEEE 8th International Conference on Knowledge and Systems Engineering (KSE 2016), Session Organier\par

\item 
情報処理学会論文誌「エンターテインメントコンピューティング」特集(2016年12月発行予定)編集委員会 編集委員\par

\item 
情報処理学会誌「音楽を軸に拡がる情報科学」特集(2016年6月号)ゲストエディタ\par

\item 
情報処理学会論文誌「音楽情報処理技術の進歩とその拡がり」特集(2016年5月号)編集委員会 幹事\par

\item 
2015年度人工知能学会全国大会(第29回) 大会委員\par

\item 
情報処理学会 第76回全国大会 プログラム編成WG 委員\par

\item 
情報処理学会/電子情報通信学会 第13回情報科学技術フォーラム(FIT 2014) プログラム委員会 委員\par

\item 
Special Session on Hot Topics in Music Information Processing, the 12th IEEE International Conference on Signal Processing, Session Co-organizer\par

\item 
日本音響学会2014年春季研究発表会 実行委員\par

\item 
情報処理学会/電子情報通信学会 第12回情報科学技術フォーラム(FIT 2013) 研究会担当委員\par

\item 
情報処理学会論文誌「音楽情報処理の新展開(音楽情報科学研究会20周年記念特集)」特集(2013年4月号)編集委員会 編集委員\par

\item 
情報処理学会 音楽情報科学研究会 幹事(2011年度〜2014年度)\par

\item 
電子情報通信学会 和文論文誌D 編集委員会 編集委員(2011年度〜2014年度)\par

\item 
情報処理学会 音楽情報科学研究会 運営委員(2007年度\&\#65374;2010年度)\par

\item 
ISMIR 2009, Local Organizing Committee Chair\par

\item 
科学技術新興機構デジタルメディア領域主催シンポジウム「表現の未来へ」推進委員(2007年度)\par

\item 
ピアノ演奏表情付けコンテスト「Rencon」Committee Member (2007年度〜2011年度)\par

\item 
論文誌査読(複数回):情報処理学会論文誌,電子情報通信学会論文誌,日本音響学会誌, 人工知能学会論文誌, IEEE Transactions on Acoustics, Speech, and Language, IEEE Journal of Selected Topics in Signal Processing, 芸術科学会, ヒューマンインタフェース学会\par

\item 
論文誌査読(1回のみ):Journal of New Music Research, Signal Processing,日本神経回路学会誌\par

\item 
国際会議論文査読:ISMIR 2007, WASPAA 2007, ISMIR 2008, ISMIR 2009, ISMIR 2010, SAPA 2010, ISMIR 2011, ACE 2015, ACE 2016, IEEE-KSE 2016\par

\item 
国内会議論文査読:FIT 2008, FIT 2009\par

\end{itemize}

\end{document}
